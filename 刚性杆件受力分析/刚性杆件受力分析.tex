\documentclass{article}
\usepackage[UTF8]{ctex}
\usepackage{amsmath,mathtools,geometry}
\geometry{scale=0.5}

\begin{document}
\fangsong 若杆件截面为边长为$a$的正方形, 面积$S=a^2$, 受轴向拉力$F$, 如图所示. 则应力
\[\sigma_0=\frac{F}{a^2}.\]
\par 若杆件受到垂直于轴线的拉力, 不妨设横截面上轴向应力的积分为拉力$F$, 且形变从截面的里侧(指垂直与轴线的拉力指向的方向)到外侧的形变量线型变化. 那么截面上从里侧到外侧的应力也为线性变化, 如图所示.
设外侧的最大应力为$\sigma_{\mathrm{max}}$, 则应力
\[\sigma(x,y)=-\frac{\sigma_{\mathrm{max}}}{a}\cdot(x-a), x,y\in[0,a].\]
那么拉力
\begin{align*}
&\iint_{S}\sigma(x,y)\mathrm{d}x\mathrm{d}y\\
=&\frac{\sigma_{\mathrm{max}}a^2}{2}\\
=&F
\end{align*}
解得
\[\sigma_{\mathrm{max}}=\frac{2F}{a^2}\]
显然$\sigma_0<\sigma_{\mathrm{max}}$, 所以轴向受力的应力更小, 材料更不容易破裂.
\end{document}