\documentclass{article}
\usepackage{amsmath,mathtools,geometry,amsfonts,amssymb}
\usepackage[UTF8]{ctex}
\geometry{a4paper,scale=0.7}
\title{}
\author{李衡岳 \and 程昊一 \and 王一丁}

\begin{document}
\maketitle
这篇文章,我们会粗略地介绍初等数论与代数数论.初等数论,是研究整数的一门数学分支;代数数论,是研究另一种整数的数学分支.初等数论被誉为数学女皇,可见她在数学之中的地位.解数论问题时,需要用的知识并不多(相对代数与几何而言),但是需要灵活的思维,因此就有极大的挑战.本篇文章我们将分为两个部分:初等数论和代数数论.

\newpage
\vspace*{\fill}
\begin{center}
	{\fontsize{42pt}{\baselineskip}\selectfont\textbf{\uppercase\expandafter{\romannumeral1}.{\CJKfamily{song}初等数论}}}
\end{center}
\vspace*{\fill}
\newpage

\newpage
\vspace*{\fill}
\begin{center}
	{\fontsize{42pt}{\baselineskip}\selectfont\textbf{\uppercase\expandafter{\romannumeral2}.{\CJKfamily{song}代数数论}}}
\end{center}
\vspace*{\fill}
\newpage
\section{复数}
\subsection{什么是复数}
求解方程是数域扩充的重要动力.例如,为了解方程$2x=1$,就需要有理数,即$\mathbb{Q}$.为了解方程$x^2=2$,就需要无理数.有理数与无理数的总称为实数,即$\mathbb{R}$.
\par 可是我们会发现,形如$x^2=-1$这样的方程就没有实数解.为此,我们引入一个新的数$\mathrm{i}$,满足$\mathrm{i}^2=-1$,叫做\textbf{虚数单位}.而且,对于$\mathrm{i}$,原有的加法和乘法的运算律仍然成立.
\par 形如$a+b\mathrm{i}(a,b\in\mathbb{R} )$的数被称为\textbf{复数},通常用字母$z$表示,即$z=a+b\mathrm{i}(a,b\in\mathbb{R})$,其中$a$被称为复数$z$的\textbf{实部},记作$\mathrm{Re}\hspace*{0.5em}z$,$b$被称为复数$z$的\textbf{虚部},记作$\mathrm{Im}\hspace*{0.5em}z$.对于复数$a+b\mathrm{i}$,当且仅当$b=0$时,它是实数;当且仅当$a=b=0$时,它是实数0;当$b\neq 0$时,叫做\textbf{虚数};当$a=0$且$b\neq 0$时,叫做\textbf{纯虚数}.
\par 全体复数构成的集合称为\textbf{复数集},记作$\mathbb{C}$.



\newpage
\begin{thebibliography}{99}
	\bibitem{ref1}
	\bibitem{ref2}
\end{thebibliography}
\end{document}