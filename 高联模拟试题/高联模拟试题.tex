\documentclass[UTF8]{ctexart}
\usepackage{amsfonts,amssymb,geometry,mathtools,amsmath}
\geometry{a4paper,scale=0.7}
\title{\vspace*{-5.5em}高中联赛一试模拟试题\vspace{-0.6em}}
\author{(时间:80分钟\quad 满分:120分)}
\date{\vspace*{-2em}}

\begin{document}
\maketitle
\hspace*{-2em}\textbf{一、填空题(共8小题,每题8分)}
\par\textbf{1.}不能表示为两个素数之和的大于2的最小偶数为\underline{\hspace{8em}}.\\
\par\textbf{2.}不能表示为$ n^2+1(n \in \mathbb{N_+}) $形式的最大素数为\underline{\hspace{8em}}.\\
\par\textbf{3.}在平面上任给$ n(n\ge 2) $个点$ A_1,A_2,\cdots ,A_n $,其中任意两点之间的距离不大于1.则\[\max{\left(\min\limits_{1\le i<j\le n} A_iA_j\right)}\]= \underline{\hspace{8em}}.\\
\par\textbf{4.}若关于$x,y,z$的方程$\dfrac{4}{n}=\dfrac{1}{x}+\dfrac{1}{y}+\dfrac{1}{z}$无正整数解,则正整数$n$的最小值为\underline{\hspace{8em}}.\\
\par\textbf{5.}记$\displaystyle \sigma(n)=\sum\limits_{d\mid n,d\in \mathbb{N_+}}d$,则满足$\sigma(n)=2n$的最小奇数为\underline{\hspace{8em}}.\\
\par\textbf{6.}已知$f(x)\in\mathbf{Z}[x]$,且$f(\mathrm{e}+\pi)=0$,则$ \mathrm{deg}\hspace{0.5em}f(x) $=\underline{\hspace{8em}}.\\
\par\textbf{7.}记$p_n$表示从小到大的第$n$个素数,则所有满足$p_{n+1}-p_n=2$的正整数$n$之和为\underline{\hspace{8em}}.\\
\par\textbf{8.}给定正整数$m$,若平面上任意$n$个点中必有$m$个点构成一个凸$m$边形的$m$个顶点,则$n$的最小值为\underline{\hspace{8em}}.\\
\\\textbf{二、解答题}
\par\textbf{9.(本题16分)}证明:
\[\int_0^1 \left( \sum\limits_{x=1}^{N}\mathrm{e}^{2\pi \mathrm{i}x^k\alpha}\right)^2 \left(\sum\limits_{x=1}^{N}\mathrm{e}^{-2\pi \mathrm{i}x^k\alpha} \right) \mathrm{d}\alpha =0.\]
\par\textbf{10.(本题20分)}对任意正整数$n$,定义:
\[f(n)=\left\{
\begin{matrix}
3n+1&,2\nmid n\\
\frac{n}{2}&,2\mid n\\
\end{matrix}
\right. .\]
对于任意给定的正整数$m$,试求最小的正整数$k$,使得$f^{(k)}(n)=1$.
\par\textbf{11.(本题20分)}记$\zeta(s)=\sum\limits_{n=1}^{+\infty}\frac{1}{n^s}\hspace{0.5em}(s\in\mathbb{C})$,证明:$\zeta(s)$的零点除负整实数外,全都具有实部$\dfrac{1}{2}$.
\end{document}