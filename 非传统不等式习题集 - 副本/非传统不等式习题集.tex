\documentclass[lang=cn,10pt]{elegantbook}
\usepackage{amsmath,mathtools,geometry}
\usepackage[fontset=none]{ctex}


\geometry{a4paper, scale=0.8}

%\newCJKfontfamily\sonti{FZCuSong-B09S}[BoldFont=FZCuSong-B09S]

\newcommand{\source}[1]{\color{second}{(\textit{#1})}\hspace{1em}}
\newcommand{\pr}[2]{
\begin{example}
\source{#1}\color{black}#2
\end{example}
\vspace{1em}
}
\newcommand{\fl}[1]{\left\lfloor #1\right\rfloor}

\title{非传统不等式习题集}

\author{lyl、chy}
\date{Sept, 2023}


% \setcounter{tocdepth}{3}

\cover{cover.jpg}

% 修改标题页的橙色带
% \definecolor{customcolor}{RGB}{32,178,170}
% \colorlet{coverlinecolor}{customcolor}

\begin{document}

\maketitle
\frontmatter



\mainmatter

\chapter*{前言}

{\kaishu “一年前, 我第一次踏入联赛考场. 那是近几年第一次代数放三, 天真的我看了一个小时后一无所获. 几个月前, 考前的我做了许多非传统的不等式问题, 再次踏入考场, 迎来的却是一道‘陈’的代数四. 我明白, 我站在了一个‘时代的边缘’……”}\par
联赛之前未了结的心愿便是整理一些这样的问题. 这里我们便整理了一些非传统的不等式问题, 以便大家学习参考.\par
上文提及的非传统不等式, 是指解答中不完全依赖代数变形与放缩技巧, 而使用了其他模块的技巧、方法或思想的不等式. 作为对比, 我们分别列出两道传统与非传统的不等式题目:\par
\textbf{传统不等式: }{\kaishu 非负实数$a_1,a_2,\cdots,a_n$满足$\sum_{i=1}^{n}a_i=n$. 求以下表达式的最大值:
\[\sum_{i=1}^n\frac{1}{1+a_i}-n\prod_{i=1}^n\frac{1}{1+a_i}.\]
}\par
\textbf{非传统不等式: }{\kaishu 设$m, n$是正整数, $x_{i,j}\in[0,1]$ $(i=1,2,\cdots,m,j=1,2,
\cdots,n)$. 求证:
\[\prod_{j=1}^n\left(1-\prod_{i=1}^m x_{i,j}\right)+\prod_{i=1}^m\left(1-\prod_{j=1}^n (1-x_{i,j})\right)\ge 1.\]}\par
虽然这个不等式可以通过数学归纳法, 运用传统方法证明, 但这个不等式的“非传统”之处在于它有一个惊人的基于概率的解法:\par
{\kaishu 构造一个$m$行$n$列的表格, 将每个格子随机地染成黑、白两种颜色. 令第$i$行、第$j$列的格子为黑色的概率为$x_{i,j}$. 则
\[\prod_{j=1}^n\left(1-\prod_{i=1}^m x_{i,j}\right)\]
表示不存在全部为黑格的一列的概率, 此事件记为$\mathrm{A}$;
\[\prod_{i=1}^m\left(1-\prod_{j=1}^n (1-x_{i,j})\right)\]
表示不存在全部为白格的一行的概率, 此事件记为$\mathrm{B}$. 由于$\overline{\mathrm{A}}$(即存在全部为黑格的一列)与$\overline{\mathrm{B}}$(即存在全部为白格的一行)不可能同时发生, 因此$\mathrm{A}$与$\mathrm{B}$之一必然发生, 故
\[\prod_{j=1}^n\left(1-\prod_{i=1}^m x_{i,j}\right)+\prod_{i=1}^m\left(1-\prod_{j=1}^n (1-x_{i,j})\right)=P(\mathrm{A})+P(\mathrm{B})\ge 1.\]}\par
这些问题往往跳脱了传统的圈套, 不再是满篇的代数变形和一些常见的通法套路, 考验我们的创新性思维. 大多数这样的问题都有着极好的选拨意义, 在很多大型比赛中, 都有它们的身影. 我们针对这样一类问题, 简单梳理了一些赛题. 该习题集按照题目来源分为若干节, 节内按照时间逆序排列. 但由于时间和精力有限, 我们未能对题目的难度进行排序, 因此可能会出现题目难度变化过大的情况, 敬请谅解. 因此若遇到实在无法解决的难题, 不妨先跳过, 等水平提升后再回头补上. 所有题目均来自于微信数之谜小程序, 题目答案也可以在数之谜相关问题中查询, 如果没有答案, 后期我们会根据自身的时间与精力在数之谜相关问题上分享解答.\par
编者保留对此习题集(PDF文件和.tex源文件)的文化产权. 此习题集仅用作学习交流使用, 请勿以盈利为目的对此习题集进行修改与转载, 编者保留对上述行为追究法律责任的权利.\par
最后, 感谢我们的老师, 如果没有他的建议, 这本习题集不会出现. 感谢在编写习题集时提出宝贵建议的所有老师和同学. 若发现本习题集的任何错误与疏漏, 或是有对于改进此习题集的建议和意见, 欢迎联系我们.\par
\quad\par
{\hfill\kaishu lyl (QQ: 1467965618)\quad chy (QQ: 487582493)}\par
{\hfill\kaishu 2023年11月17日}

\tableofcontents

\chapter*{题目}

\section{全国高中数学联赛}

\pr{2022 全国高中数学联赛-A卷加试 P3}{设 $a_1,a_2,\cdots,a_{100}$ 是非负整数, 同时满足以下条件:
\begin{enumerate}
\item 存在正整数 $k\leq100$,使得 $a_1\leq a_2\leq\cdots\leq a_k$ , 而当 $i>k$ 时 $a_i=0$;
\item $\:a_1+a_2+a_3+\cdots+a_{100}=100$;
\item $a_1+2a_2+3a_3+\cdots+100a_{100}=2022$.
\end{enumerate}
求 $a_1+2^2a_2+3^2a_3+\cdots+100^2a_{100}$ 的最小可能值.}

\pr{2023 全国高中数学联赛-A卷加试 P3}{求具有下述性质的最小正整数$k$: 若将 $1,2,\cdots,k$ 中的每个数任意染为红色或者蓝色, 则或者存在 9 个互不相同的红色的数 $x_1,x_2,\cdots,x_9$ 满足$x_1+x_2+\cdots+x_8<x_9$, 或者存在 10 个互不相同的蓝色的数 $y_1,y_2,\cdots,y_{10}$ 满足$y_1+y_2+\cdots+y_9<y_{10}$.}

\section{中国数学奥林匹克}
\pr{2019 CMO P1}{设实数$a_1,a_2,\cdots,a_{40}$满足$\sum_{i=1}^{40}a_i=0$, 且对$1\leq i\leq40$,都有$|a_i-a_{i+1}|\leq1$,这里 $a_{41}=a_1$. 记 $a=a_{10},b=a_{20},c=a_{30},d=a_{40}.$
\begin{enumerate}
    \item 求 $a+b+c+d$ 的最大值;
    \item 求 $ab+cd$ 的最大值.
\end{enumerate}}

\pr{2015 CMO P1}{设正整数 $a_1,a_2,\cdots,a_{31},b_1,b_2,\cdots,b_{31}$ 满足: 
\begin{enumerate}
    \item $a_1<a_2<\cdots<a_{31}\leq2015$, $b_1<b_2<\cdots<b_{31}\leq2015$;
    \item $a_1+a_2+\cdots+a_{31}=b_1+b_2+\cdots+b_{31}.$
\end{enumerate}
求 $S=|a_1-b_1|+|a_2-b_2|+\cdots+|a_{31}-b_{31}|$ 的最大值.}

\pr{1995 CMO P5}{设 $a_1,a_2,\cdots,a_{10}$ 是不同的正整数, 和为 1995.求
\[a_1a_2+a_2a_3+\cdots+a_{10}a_1\]
的最小值.}

\section{北方希望之星数学夏令营}

\pr{2021 北方希望之星数学夏令营 P5}{求最大的正整数 $n$, 使得存在 $n$ 个正整数$x_1<x_2<\cdots<x_n$, 满足
\[x_1+x_1x_2+\cdots+x_1x_2\cdots x_n=2021.\]
}

\section{中国国家集训队选拔}

\pr{2023 CTST P11}{ 设 $n$ 是正整数, $a_{ijk}\in\{-1,1\}(1\leq i,j,k\leq n)$ . 求证: 存在 $x_1,\cdots,x_n,y_1,\cdots,y_n,z_1,\cdots,z_n\in\{-1,1\}$, 使得
\[\left|\sum_{s=1}^{n}a_{ijk}x_{i}y_{j}z_{k}\right|>\frac{n^{2}}{3}.\]
}

\pr{2023 CTST P14}{对非空有限实数集 $B$ 和实数 $x$, 定义
\[d_B(x)=\min_{b\in B}|x-b|.\]
\begin{enumerate}
	\item 给定正整数 $m$. 求最小的实数 $\lambda$, 使得对任意正整数 $n$ 和任意实数 $x_1,x_2,\cdots,x_n\in[0,1]$, 都存在 $m$ 元实数集 $B$, 满足
	\[d_B(x_1)+d_B(x_2)+\cdots+d_B(x_n)\leq\lambda n.\]
	\item 设 $m$ 是正整数, $\varepsilon$ 是正实数. 求证: 存在正整数 $n$ 和非负实数 $x_1,x_2,\cdots,x_n$, 满足对任意 $m$ 元实数集$B$, 都有
	\[d_B(x_1)+d_B(x_2)+\cdots+d_B(x_n)>(1-\varepsilon)(x_1+x_2+\cdots+x_n).\]
\end{enumerate}}

\pr{2023 CTST P21}{ 给定整数 $n\geq2$. 求最小的实数 $\lambda$, 使得对任意实数$a_1,a_2,\cdots,a_n$ 及 $b$, 均有
\[\lambda\sum_{i=1}^n\sqrt{|a_i-b|}+\sqrt{n\left|\sum_{i=1}^na_i\right|}\geq\sum_{i=1}^n\sqrt{|a_i|}.\]}

\pr{2022 CTST P9}{设 $a_1,a_2,\cdots,a_n$ 是 $n$ 个两两互相不整除的正整数, 求证:
\[a_1+a_2+\cdots+a_n\geq1.1n^2-2n.\]}

\pr{2022 CTST P10}{给定正整数 $n$. 求使 $\mathbb{R}^n$ 上的函数
\[f(x_1,x_2,\cdots,x_n)=\sum_{k_1=0}^2\sum_{k_2=0}^2\cdots\sum_{k_n=0}^2|k_1x_1+k_2x_2+\cdots+k_nx_n-1|\]
达到最小值时的所有 $(x_1,x_2,\cdots,x_n)$.}

\pr{2022 CTST P12}{设整数 $m\geq n\geq2022,\quad a_1,a_2,\cdots,a_n,b_1,b_2,b_n$ 是实数. 求证: 使得 $|a_i+b_j-ij|\leq m$ 的 $(i,j)\ (1\leq i,j\leq n)$的对数不超过 $3n\sqrt{m\ln n}$.}

\pr{2022 CTST P23}{设 $n$ 是正整数, $2n$ 个非负实数 $x_1,x_2,\cdots,x_{2n}$ 满足$x_1+x_2+\cdots+x_{2n}=4$ . 求证: 存在非负整数 $p,q$ , 使得 $q\leq n-1$, 且
\[\sum_{i=1}^qx_{p+2i-1}\leq1,\ \sum_{i=q+1}^{n-1}x_{p+2i}\leq1,\]
其中下标按模 $2n$ 理解.}

\pr{2021 CTST P11}{设$n$是正整数, $a_1,a_2,\cdots,a_n$是正实数. 对$1\le k\le 2n+1$, 记
\[b_k=\max\limits_{0\le m\le n}\left(\frac{1}{2m+1}\sum_{i=k-m}^{k+m}a_i\right),\]
其中角标按模$2n+1$理解. 求证: 满足$b_k\ge 1$的$k$的个数不超过$2\sum_{i=1}^{2n+1}a_i$.}

\pr{2021 CTST P15}{求最大的实数 $C$, 使得对任意整数 $n\geq2$, 存在 $x_1,x_2,\cdots,x_n\in[-1,1]$, 满足
\[\prod_{1\leq i<j\leq n}\left(x_i-x_j\right)\geq C^{n(n-1)/2}.\]}

\pr{2021 CTST P19}{给定整数 $n\geq2$. 求最小的正整数 $m$, 使得存在不同的正实数 $x_{ij}(1\leq i,j\leq n)$,满足:
\begin{enumerate}
    \item 对任意 $1\leq i,j\leq n,$
\[x_{ij}=\max\{x_{i1},x_{i2},\cdots,x_{ij}\}\]
或
\[x_{ij}=\max\{x_{1j},x_{2j},\cdots,x_{ij}\};\]
\item 对任意 $1\leq i\leq n$,至多有 $m$ 个脚标 $k$ 使得$x_{ik}=\max\{x_{i1},x_{i2},\cdots,x_{ik}\};$
\item 对任意 $1\leq j\leq n$,至多有 $m$ 个脚标 $k$ 使得
 $x_{kj}=\max\{x_{1j},x_{2j},\cdots,x_{kj}\}$ .
\end{enumerate}}

\pr{2020 CTST P5}{设 $n$ 是正整数, $a_1,a_2,\cdots,a_n$ 是 $1,2,\cdots,n$ 的一个排列, 求
\[\sum_{i=1}^n\min\{a_i,2i-1\}\]
的最小值.}

\pr{2018 CTST P6}{设$m, n$为正整数, $A_1,A_2,\cdots,A_m$是某个$n$元集合的$m$个子集. 求证:
\[\sum_{i=1}^m\sum_{j=1}^m|A_i|\cdot|A_i\cap A_j|\ge\frac{1}{mn}\left(\sum_{i=1}^m|A_i|\right)^3.\]}

\pr{2017 CTST P2}{设实数$x>1$, $n$是正整数. 求证:
\[\sum_{k=1}^n\frac{\{kx\}}{\fl{kx}}<\sum_{i=1}^n\frac{1}{2k-1}.\]}

\pr{2017 CTST P21}{求满足以下条件的数组 $(x_1,x_2,\cdots,x_{100})$ 的个数: 
\begin{enumerate}
    \item $x_1,x_2,\cdots,x_{100}\in\{1,2,\cdots,2017\}$ ;
    \item $2017\mid x_1+x_2+\cdots+x_{100}$ ;
    \item $2017\mid x_1^2+x_2^2+\cdots+x_{100}^2$.
\end{enumerate}}

\pr{2013 CTST P4}{
设 $n,k$ 为给定的大于 1 的整数, 非负实数 $a_1,a_2,$ $\cdots,a_n;c_1,c_2,\cdots,c_n$ 满足
\begin{enumerate}
    \item $a_1\geq a_2\geq\cdots\geq a_n$ ,且$a_1+a_2+\cdots+a_n=1$ ;
    \item 对 $m=1,2,\cdots,n$ ,有$c_1+c_2+\cdots+c_m\leq m^k.$
\end{enumerate}
求 $c_1a_1^k+c_2a_2^k+\cdots+c_na_n^k$ 的最大值.}

\pr{2009 CTST P23}{设 $m$ 是大于 1 的整数, $n$ 是奇数且 $3\leq n<2m$.数$a_{i,j}(1\leq i\leq m,1\leq j\leq n)$ 满足:
\begin{enumerate}
    \item 对任意 $1\leq j\leq n,\quad a_{1,j},a_{2,j},\cdots,a_{m,j}$ 是 $1,2,\cdots,m$ 的一个排列;
    \item 对任意 $1\leq i\leq m,1\leq j\leq n-1,$ $|a_{i,j}-a_{i,j+1}|\leq1$.
\end{enumerate}
求 
\[M=\max\limits_{1\leq i\leq m}\sum\limits_{j=1}^na_{i,j}\]
的最小值.}

\pr{2008 CTST P24}{求最大的正实数 $M$,使得对任意正整数 $n$, 存在正实数数列 $a_1,a_2,\cdots,a_n$ 及 $b_1,b_2,\cdots,b_n$, 满足:
\begin{enumerate}
    \item $\sum_{k=1}^n b_k=1$,且对 $2\leq k\leq n-1$, $2b_k\geq b_{k-1}+b_{k+1}$ ;
    \item 对 $1\leq k\leq n$ , \[a_k^2\leq1+\sum\limits_{i=1}^ka_ib_i,\]且 $a_n=M.$
\end{enumerate}}

\pr{2007 CTST P23}{设 $x_1,x_2,\cdots,x_n$ 是不全相等且和不等于 0 的实数, 记
\[A=\left|\sum_{i=1}^nx_i\right|,\quad B=\max_{1\leq i<j\leq n}\lvert x_i-x_j\rvert.\]
求证: 对任意 $n$ 个复数 $z_1,z_2,\cdots,z_n$ ,存在 $x_1,x_2,$ $\cdots,x_n$ 的一个排列 $y_1,y_2,\cdots,y_n$,使得
\[\left|\sum_{i=1}^ny_iz_i\right|\geq\frac{AB}{2A+B}\max_{1\leq i\leq n}\lvert z_i\rvert.\]
}

\pr{2006 CTST P3}{
设 $a_1,a_2,\cdots,a_n$ 是 $n$ 个实数.求证: 存在实数 $b_1,b_2,$ $\cdots,b_n,$ 满足:
\begin{enumerate}
    \item 对 $1\leq i\leq n,\quad a_i-b_i$ 是正整数;
    \item \[\sum_{1\leq i<j\leq n}(b_i-b_j)^2\leq\frac{n^2-1}{12}.\]
\end{enumerate}}

\section{中国女子数学奥林匹克}

\pr{2022 CGMO P1}{考虑所有满足以下两个条件的实数序列 $x_0,x_1,x_2,$ $\cdots,x_{100}:$
\begin{enumerate}
    \item $x_0=0$;
    \item 对任意 $1\leq i\leq100,$ 有 $1\leq x_i-x_{i-1}\leq2.$
\end{enumerate}
求最大的正整数 $k\leq100$, 使得对任意这样的序列, 均有
\[x_k+x_{k+1}+\cdots+x_{100}\geq x_0+x_1+\cdots+x_{k-1}.\]}

\pr{2020 CGMO P2}{给定整数 $n\geq2$.设 $x_1,x_2,\cdots,x_n$ 是实数, 求
\[2\sum_{1\le i<j\le n}\lfloor x_ix_j\rfloor-(n-1)\sum_{i=1}^{n}\lfloor x_i^2\rfloor\]
的最大值.}

\pr{2017 CGMO P5}{求最大的实数 $c$, 使得对任意正整数 $n$ 及任意满足$0=x_0<x_1<\cdots<x_n=1$的数列 $\{x_n\},$ 都有 
\[\sum_{k=1}^nx_k^2(x_k-x_{k-1})>c.\]}

\pr{2014 CGMO P2}{
给定整数 $n\geq2$.设 $x_1,x_2,\cdots,x_n$ 是正实数且满足$\fl{x_1},\fl{x_2},\cdots,\fl{x_n}$ 是 $1,2,\cdots,n$ 的一个排列.求
\[\sum_{i=1}^{n-1}\lfloor x_{i+1}-x_i\rfloor \]
的最大值和最小值.}

\section{西部数学邀请赛}

\pr{2019 CWMO P4}{给定整数 $n\geq2.$ 求最小的实数$\lambda$, 使得对任意实数 
$x_1,x_2,\cdots,x_n\in[0,1]$,存在 $\varepsilon_1,\varepsilon_2,\cdots,\varepsilon_n$ $\in\{0,1\}$,满足对任意 $1\leq i\leq j\leq n$, 都有
\[\left|\sum_{k=i}^j(\varepsilon_k-x_k)\right|\leq\lambda.\]}

\pr{2018 CWMO P1}{设实数 $x_1,x_2,\cdots,x_{2018}$ 满足: 对任意$1\leq i<j\leq2018$,均有 $x_i+x_j\geq(-1)^{i+j}$.求
\[\sum_{i=1}^{2018} ix_i\]
的最小值.}

\pr{2017 CWMO P2}{若存在正整数 $x_1,x_2,\cdots,x_n$ 满足
\[x_1x_2\cdots x_n(x_1+x_2+\cdots+x_n)=100n,\]
求正整数 $n$ 的最大值.}

\pr{2015 CWMO P1}{给定正整数 $n$.设实数 $x_1,x_2,\cdots,x_n$ 满足 $\sum_{i=1}^{n} x_i$ 为
整数.记 
\[d_k=\min_{m\in\mathbb{Z}}|x_k-m|,1\leq k\leq n,\]
求 $\sum_{k=1}^{n}d_k$的最大值.}

\pr{2014 CWMO P6}{给定整数 $n\geq2.$ 设实数 $x_1,x_2,\cdots,x_n$ 满足:
\begin{enumerate}
    \item $\sum_{i=1}^{n} x_i=0$;
    \item $\left|x_i\right|\leq1,1\leq i\leq n$
\end{enumerate}
求$\min\limits_{1\leq i\leq n-1}|x_i-x_{i+1}|$的最大值.}

\pr{2009 CWMO P8}{给定整数 $n\geq3$.设 $a_1,a_2,\cdots,a_n$ 满足$a_1+a_2+\cdots+a_n=0$,且对 $2\leq k\leq n-1,$
 $2a_k\leq a_{k-1}+a_{k+1}$.求最小的实数 $\lambda$, 使得对任意
 $k\in\{1,2,\cdots,n\}$,都有
 \[|a_k|\leq\lambda\max\{|a_1|,|a_n|\}.\]}

\pr{2008 CWMO P8}{设 $P$ 为正 $n$ 边形 $A_1A_2\cdots A_n$ 内的任意一点.对
 $1\leq i\leq n$,直线 $A_iP$ 交正 $n$ 边形 $A_1A_2\cdots A_n$ 的边界于另一点 $B_i$.求证:
\[\sum_{i=1}^{n}PA_i\ge\sum_{i=1}^{n}PB_i.\]}

\pr{2003 CWMO P2}{设实数 $a_1,a_2,\cdots,a_{2n}$ 满足
\[\sum_{i=1}^{2n-1}(a_{i+1}-a_i)^2=1,\]
求
\[(a_{n+1}+a_{n+2}+\cdots+a_{2n})-(a_1+a_2+\cdots+a_n)\]
的最大值.}

\section{中国东南地区数学奥林匹克}
{\kaishu 题号“$1\,\text{-}\,x$”表示高一组的第$x$题, “$2\,\text{-}\,x$”表示高二组的第$x$题.}

\pr{2023 CSMO P1-7}{称正整数 $S$ 为"育英数", 如果存在正整数 $n$ 以及$2n$ 个正整数 $a_1,a_2,\cdots,a_n,b_1,b_2,\cdots,b_n$, 使得
\[S=\sum_{i=1}^na_ib_i,\ \sum_{i=1}^n(a_i^2-b_i^2)=1,\ \sum_{i=1}^n(a_i+b_i)=2023.\]
求: (1)最小的育英数; (2)最大的育英数.}

\pr{2020 CSMO P1-4}{设 $a_1,a_2,\cdots,a_{17}$ 是 $1,2,\cdots,17$ 的一个排列, 且满足
\[(a_1-a_2)(a_2-a_3)\cdots(a_{16}-a_{17})(a_{17}-a_1)=2^n.\]
正整数 $n$ 的最大值.}

\pr{2017 CSMO P1-4}{设实数 $a_1,a_2,\cdots a_{2017}$满足 $a_1=a_{2017},$
\[|a_i+a_{i+2}-2a_{i+1}|\leq1\ (i=1,2,\cdots,2015).\]
求$\max\limits_{1\le i<j\le 2017}|a_i-a_j|$的最大值.}

\section{陈省身杯全国高中数学奥林匹克}

\pr{2021 陈省身杯 P2}{给定整数 $n\geq2.$ 实数 $x_1,x_2,\cdots,x_n$ 满足
\[\min_{1\leq k\leq n}\left\{\frac{x_1+x_2+\cdots+x_k}k\right\}=0,\quad\max_{1\leq k\leq n}\left\{\frac{x_1+x_2+\cdots+x_k}k\right\}=1.\]
记
\[M=\max_{1\leq i<n}\{x_i\},m=\min_{1\leq i\leq n}\{x_i\},\]
求 $M-m$ 的最小值和最大值.}

\section{希望联盟夏令营}

\pr{希望联盟夏令营 2023-3 P13}{设 $a_1,a_2,\cdots,a_{100}$ 是和为 1000 的 100 个正整数. 记\[S=a_1a_2+a_2a_3+\cdots+a_{99}a_{100}.\]
求 $S$ 的最大值, 并确定使 $S$ 取到最大值的所有可能数组 $(a_1,a_2,\cdots,a_{100})$ 的个数.} 

\pr{希望联盟夏令营 2021-1 P11}{已知正实数 $a_1,a_2,\cdots,a_{2022},b_1,b_2,\cdots,b_{2022}$ 满足
\[a_1+a_2+\cdots+a_{2022}=b_1+b_2+\cdots+b_{2022}=1,\]
求
\[S=\min_{1\leq i\leq2022}\frac{a_i}{b_i}+\min_{1\leq i\leq2022}\frac{b_i}{a_i}+\sum_{i=1}^{2022}\lvert a_i-b_i\rvert\]
的最大值.}

\pr{希望联盟夏令营 2021-3 P12}{设 $a_1,a_2,\cdots,a_{2021}$ 是整数, 满足$1=a_1\leq a_2\leq\cdots\leq a_{2021}=100$ .记
\[f=(a_1^2+a_2^2+\cdots+a_{2021}^2)-(a_1a_3+a_2a_4+\cdots+a_{2019}a_{2021}).\]
求$f$ 的最大值 $f_0$, 并求使得 $f=f_0$ 成立的数组 $(a_1,a_2,\cdots,a_{2021})$ 的个数.}

\pr{希望联盟夏令营 2021-2 P4}{设 $\vec{u}=(u_1,u_2,u_3)$ 和 $\vec{v}=(v_1,v_2,v_3)$ 是空间向量, 满足 $u_i,v_i\left(i=1,2,3\right)$ 均为整数, 且
$0.9999<\cos\langle\vec{u},\vec{v}\,\rangle<1.$ 记
\[S=|u_1|+|u_2|+|u_3|+|v_1|+|v_2|+|v_3|, \]
求$\fl{\sqrt{S}}$的最小可能值.}

\pr{希望联盟夏令营 2020-1 P12}{没正实数 $a_1,a_2,\cdots,a_{2020}$ 满足 $\sum_{i=1}^{2020} a_i=2020$, 求
\[\sum_{k=1}^{2020}a_k^{1/k^2}\]
的最大值.}

\pr{希望联盟夏令营 2020-2 P13}{设整数 $n\geq2,\quad x_1,x_2,\cdots,x_n$ 为互不相同的正实数。求证: 可以选取 $a_1$, $a_2$, $\cdots$, $a_n\in\{-1,1\}$, 使得
\[\sum_{i=1}^na_ix_i^2>\left(\sum_{i=1}^na_ix_i\right)^2.\]}

\pr{希望联盟夏令营 2019-2 P13}{给定正整数 $n$.求最小的实数 $\lambda$, 使得存在区间 $[0,1]$ 内的实数 $a_1,a_2,\cdots,a_n$, 满足对任意
 $0\leq x_1\leq x_2\leq\cdots\leq x_n\leq1$,均有$\min\limits_{1\leq i\leq n} |x_i-a_i|\leq\lambda_.$}

\section{新星数学奥林匹克}

\pr{新星数学奥林匹克 2023春季 P2}{给定偶数 $n\geq4$.实数 $a_1\geq a_2\geq\cdots\geq a_n\geq0$ 满足
\[a_1+a_3+\cdots+a_{n-1}=3,\ \quad a_2+a_4+\cdots+a_n=1.\]
求$\sum_{i=1}^na_i^2$的最小值.}

\pr{新星数学奥林匹克 2022春季 P2}{设非负实数 $a_1,a_2,\cdots,a_{15}$ 满足$a_1+a_2+\cdots+a_{15}=1$.求
\[\sum_{1\le i<j\le 15}\left(\frac32\right)^{i+j}a_ia_j\]
的最大值.}

\pr{新星数学奥林匹克 2021春季 P3}{给定整数 $n\geq4.$ 求最大的实数 $\lambda$, 使得对任意满足$\sum_{i=1}^na_i=1$ 的非负实数 $a_1,a_2,\cdots,a_n$, 均有
\[\sum_{i=1}^na_ia_{i+1}\leq\frac14-\lambda mM,\]
其中 $a_{n+1}=a_1,\ m=\min\{a_1,a_2,\cdots,a_n\},\ M=\max\{a_1,a_2,\cdots,a_n\}$.}

\pr{新星数学奥林匹克 2020秋季 P1}{求最小的实数$\lambda$,使得对任意满足$\sum_{i=1}^ {20} ix_i=0$ 的实数$x_1,x_2,\cdots,x_{20}$,都有
\[\left|\sum_{i=1}^{20}i^2x_i\right|\leq\lambda\max_{1\leq i\leq20}|x_i|.\]}

\pr{新星数学奥林匹克 2019夏季 P1}{设 $x_1,x_2,\cdots,x_{2019}$ 是实数,满足$x_1+x_2+\cdots+x_{2019}\in\mathbb{Z}.$求
\[\sum_{1\leq i<j\leq2019}\left\{x_i+x_j\right\}\]
的最大值, 其中$\{x\}$为$x$的小数部分.}

\pr{新星数学奥林匹克 2018秋季 P3}{给定整数 $n\geq2.$ 设 $x_1,x_2,\cdots,x_n$ 是正实数, 满足对任意 $1\leq i<j\leq n$ 都有 $x_ix_j\geq i$.求 $x_1x_2\cdots x_n$的最小值.}

\pr{新星数学奥林匹克 2018春季 P1}{给定整数 $n\geq2$.设非负实数 $x_1,x_2,\cdots,x_n$ 满足$\sum_{i=1}^nx_i=n$.求
\[\left(\sum_{i=1}^{n}\fl{x}\right)\left(\sum_{i=1}^{n}\{x\}\right)\]
的最大值.}

\pr{新星数学奥林匹克 2017秋季 P4}{给定整数 $n\geq2$.求最小的实数 $c$, 使得对任意非负实数$a_1,a_2,\cdots,a_n$,都存在 $i\in\{1,2,\cdots,n\}$, 满足$a_{i-1}+a_{i+1}\leq ca_i$ ,其中 $a_0=a_{n+1}=0$.}

\pr{新星数学奥林匹克 2016秋季 P1}{设 $x_1,x_2,\cdots,x_n$ 是 $n$ 个不同的实数, 记$D=\max_{1\leq i<j\leq n}|x_i-x_j|$。求证: 存在 $x_1,x_2,\cdots,x_n$
的一个排列 $y_1,y_2,\cdots,y_n$, 使得
\[\left|\sum_{i=1}^{n}iy_i\right|\ge\frac{n-1}{2}D.\]}

\pr{新星数学奥林匹克 2017夏季 P1}{给定整数 $n\geq2$.设实数 $a_1,a_2,\cdots,a_n$ 满足
\[\sum_{i=1}^n\lvert a_i\rvert+\left\lvert\sum_{i=1}^na_i\right\rvert=1.\]
求$\sum_{i=1}^{n}a_i^2$的最小值和最大值.}

\section{学而思数学竞赛联考}

\pr{XMO 12th P13}{已知 $a_1,a_2,a_{22}\in[1,2]$, 令$a_{23}=a_1$, 求
\[{\left(\sum_{i=1}^{22}a_ia_{i+1}\right)}\left/ {\left(\sum_{i=1}^{22}a_i\right)^2}\right.\]的最大值.}

\section{土耳其数学奥林匹克第二轮}

\pr{2022 Turkey MO Round 2 P3}{设 $a_1,a_2,\cdots,a_{2022}$是非负实数,满足 $a_1+a_2+\cdots+a_{2022}=1$ .求数对 $(i,j)$ 个数的最大值,满足 $1\leq i,j\leq2022$,且 $a_i^2+a_j\geq\frac{1}{2021}$.}

\pr{2019 Turkey MO Round 2 P5}{设函数 $f:\{1,2,\cdots,2019\}\to\{-1,1\}$, 满足对任意 $1\leq k\leq2019$, 存在$1\leq l\leq2019$, 使得
\[\sum_{i{:}(l-i)(i-k)\ge 0}f(i)\le 0.\]
求$\sum_{i=1}^{2019}f(i)$的最大值.}

\section{谜之竞赛}

\pr{谜之竞赛 2023年7月 P13}{设数列 $\{a_n\}$ 满足 $a_1=1$, 且对任意正整数 $n$, $a_{n+1}=a_n+a_{i_n}$ ,其中
\[i_n=\min\{1\leq i\leq n\mid3a_i+2023\geq a_n\}.\]
求证: 存在非负整数 $d,N$, 使得对任意整数 $n\geq N$, 都有 $a_{n+1}=a_n+a_{n-d}$.}

\section{保加利亚数学奥林匹克}

\pr{Bulgaria MO 2023 P5}{给定正整数 $n$.设实数 $x_1,x_2,\cdots,x_n$ 满足$|x_1|+|x_2|+\cdots+|x_n|=1$,求
\[|x_1|+|x_1-x_2|+|x_1+x_2-x_3|+\cdots+|x_1+x_2+\cdots+x_{n-1}-x_n|\]
的最小值.}

\pr{Bulgaria MO 2020 P2}{设 $n$ 是正整数, 非负实数 $b_1,b_2,\cdots,b_n$ 的和为$2$, 实数 $a_0,a_1,\cdots,a_n$ 满足 $a_0=a_n=0$ , 且对任意$1\leq i\leq n$, $|a_i-a_{i-1}|\leq b_i$. 求证:
\[\sum_{i=1}^n(a_i+a_{i-1})b_i\leq2.\]}

\pr{Bulgaria MO 2018 P3}{求证:
\[\left(\frac65\right)^{\sqrt{3}}>\left(\frac54\right)^{\sqrt{2}}.\]}

\section{韩国数学奥林匹克决赛}

\pr{Korea MO Final 2013 P3}{设整数 $n\geq2$, 记集合
\[T=\{(i,j)\mid1\leq i<j\leq n,i\mid j\}.\]
设 $x_1,x_2,\cdots,x_n$ 是非负实数, 满足$x_1+x_2+\cdots+x_n=1$.求 $\sum\limits_{(i,j)\in T} x_ix_j$ 的最大值.}

\section{国际数学奥林匹克预选题代数}

\pr{2022 IMOSL A4}{设整数 $n\geq3$, 实数$x_1,x_2,\cdots,x_n\in[0,1]$. 记$s=x_1+x_2+\cdots+x_n$ , 且设 $s\geq3$.求证: 存在$1\leq i<j\leq n$,使得
\[2^{j-i}x_ix_j>2^{s-3}.\]}

\pr{2022 IMOSL A5}{求所有的整数 $n\geq2$, 使得存在正实数$a_1<a_2<\cdots<a_n$ 和正实数 $r$, 满足
\[\{a_j-a_i\mid1\leq i<j\leq n\}=\{r,r^2,\cdots,r^{n(n-1)/2}\}.
\]}

\pr{2021 IMOSL A1}{设 $n$ 是正整数, $A$ 是 $\{0,1,\cdots,5^n\}$ 的一个$4n+2$ 元子集. 求证: 存在 $A$ 中的元素 $a<b<c$ , 使得$c+2a>3b$.}

\pr{2021 IMOSL A2}{求所有的正整数$n$, 使得
\[\sum_{i=1}^n\sum_{j=1}^n\fl{\frac{ij}{n+1}}=\frac{n^2(n-1)}4.\]}

\pr{2021 IMOSL A3}{给定正整数 $n$.设 $a_1,a_2,\cdots,a_n$ 是 $1,2,\cdots,n$的一个排列, 求下列表达式的最小值:
\[\sum_{k=1}^{n}\fl{\frac{a_k}{k}}.\]}

\pr{2020 IMOSL A1}{给定正整数 $N$. 求最小的实数 $b_N$, 使得对任意实数$x$, 均有
\[\sqrt[N]{\frac{x^{2N}+1}2}\leq b_N(x-1)^2+x.\]}

\pr{2019 IMOSL A2}{设实数 $u_1,u_2,\cdots,u_2019$, 满足
\[\sum_{i=1}^{2019}u_i=0,\ \sum_{i=1}^{2019}u_i^2=1.\]
设$a=\min\{u_1,u_2,\cdots,u_{2019}\}$, $b=\max\{u_1,u_2,\cdots,u_{2019}\}$. 求证: $ab<-1/2019.$}

\pr{2019 IMOSL A3}{设整数$n\geq3$, $a_1,a_2,\cdots,a_n$ 是和为 2 的严格递增的正实数数列. $X$ 是集合 $\{1,2,\cdots,n\}$ 的子集, 使得
\[\left|1-\sum_{i\in X}a_i\right|\]
最小. 求证: 存在和为$2$的严格递增的正实数数列 $b_1,b_2,\cdots,b_n$, 使得
\[\sum_{i\in X}b_i=1.\]}

\pr{2019 IMOSL A4}{设整数 $n\geq2$, 实数 $a_1,a_2,\cdots,a_n$ 满足$a_1+a_2+\cdots+a_n=0$.定义集合
\[A=\{(i,j):\ 1\leq i<j\leq n,|a_i-a_j|\geq1\}.\]
求证:若 $A$ 非空,则
\[\sum_{(i,j)\in A}a_ia_j<0.\]}

\pr{2018  IMOSL A3}{设 $S$ 是由正整数构成的集合, 求证: 下述命题中至少有一个成立:
\begin{enumerate}
	\item 存在 $S$ 的不同的有限子集 $F,G$, 使得
	\[\sum_{x\in F}\frac1x =\sum_{x\in G}\frac1x ;\]
	\item 存在有理数 $r\in(0,1)$, 使得对 $S$ 的任一有限子集$F$, 
	\[\sum_{x\in F}\frac1x \neq r.\]
\end{enumerate}}

\pr{2018 IMOSL A4}{设数列$\{a_n\}$满足$a_0=0,a_1=1$, 且当$n\geq2$时, 存在$1\leq k\leq n$, 使得
\[a_n=\frac{a_{n-1}+a_{n-2}+\cdots+a_{n-k}}k.\]
求 $a_{2018}-a_{2017}$ 的最大可能值.}

\pr{2017 IMOSL A2}{设$q$ 是实数. 甲有一张餐巾纸, 上面写着 10 个不同的实数, 他在黑板上写下以下三行实数:\par
在第一行, 甲写下了所有形如 $a-b$ 的数, 其中 $a,b$是餐巾纸上的两个数 (可以相同);\par
在第二行, 甲写下了所有形如 $qab$ 的数, 其中 $a,b$ 是第一行的两个数 (可以相同);\par
在第三行, 甲写下了所有形如 $a^2+b^2-c^2-d^2$ 的数, 其中 $a,b,c,d$ 是第一行中的四个数 (可以相同).\par
求所有的 $q$, 使得无论餐巾纸上的数是什么, 第二行中的每个数都出现在第三行中.}

\pr{2017 IMOSL A5}{给定整数 $n\geq3.$ 设 $x_1,x_2,\cdots,x_n$ 是实数, 如果对它的任意一个排列 $y_1,y_2,\cdots,y_n$, 都有
\[\sum_{i=1}^{n-1}y_iy_{i+1}\ge -1,\]
求最大的实数$\lambda$, 使得总有
\[\sum_{1\le i<j\le n}x_ix_j\ge\lambda.\]}

\pr{2016 IMOSL A2}{求最小的实数 $C$, 使得对任意正实数 $a_1,a_2,a_3,a_4,a_5$(允许相同), 总可以选择不同的下标 $i,j,k,l$, 满足
\[\left|\frac{a_i}{a_j}-\frac{a_k}{a_l}\right|\le C.\]}

\pr{2016 IMOSL A3}{求所有的整数 $n\geq3$, 使得对任意满足$|a_k|+|b_k|=1$ $(1\leq k\leq n)$的实数 $a_1,a_2,\cdots,a_n$和 $b_1,b_2,\cdots,b_n$, 均存在 $\varepsilon_1,\varepsilon_2,\cdots,\varepsilon_n\in\{-1,1\}$ 满足
\[\left|\sum_{k=1}^n\varepsilon_ka_k\right|+\left|\sum_{k=1}^n\varepsilon_kb_k\right|\leq1.\]}

\pr{2016 IMOSL A8}{求最大的实数 $a$, 使得对任意正整数 $n$ 及任意实数$0=x_0<x_1<\cdots<x_n$ , 均有
\[\sum_{i=1}^n\frac1{x_i-x_{i-1}}\geq a{\sum_{i=1}^n}\frac{i+1}{x_i}.\]}

\pr{2015 IMOSL A3}{给定正整数$n$. 设$x_1,x_2,\cdots,x_{2n}\in[-1,1]$. 求下列表达式的最大值:
\[\sum_{1\le r<s\le 2n}(s-r-n)x_rx_s.\]}

\pr{2014 IMOSL A2}{定义函数
\[f(x)=
\begin{cases}
x+\frac12, \ 0<x<\frac12;\\
x^2,\quad\ \ \,\frac12\le x<1.
\end{cases}\]
设正实数 $a,b$ 满足 $a<b<1$ ,数列 $\{a_n\},\{b_n\}$满足$a_0=a,b_0=b,a_n=f(a_{n-1}),~b_n=f(b_{n-1}),n\ge1$. 求证: 存在正整数$n$, 满足
\[(a_n-a_{n-1})(b_n-b_{n-1})<0.\]}

\pr{2014 IMOSL A3}{对于实数数列 $x_1,x_2,\cdots,x_n$, 定义其“价值”为
\[\max_{1\leq i\leq n}\{|x_1+x_2+\cdots+x_i|\}.\]\par
给定 $n$ 个实数, 甲和乙想把这 $n$ 个实数排成低价值的数列.一方面, 勤奋的甲检验了所有可能的方式来寻找其最小的价值 $D$.另一方面, 贪婪的乙先选择 $x_1$, 使得 $|x_1|$ 尽可能地小; 再在剩下的数中选择 $x_2$, 使得$|x_1+x_2|$ 尽可能地小; $\cdots$; 在第 $i$ 步, 在剩下的数中选择 $x_i$, 使得 $|x_1+x_2+\cdots+x_i|$ 尽可能地小在每一步, 若有不止一种选择, 则乙任意选择一种. 设乙最后得到的数列的价值为$G$.\par
求最小的实数 $c$, 使得对于每个正整数 $n$、每个由 $n$ 个实数构成的数组和每个乙可以得到的数列, 均有
\[G\le cD.\]}

\pr{2013 IMOSL A2}{求证: 在任意由 $2000$ 个不同实数构成的集合中, 存在实数 $a>b$ 和 $c>d$, 使得 $a\neq c$ 或 $b\neq d$, 且
\[\left|\frac{a-b}{c-d}-1\right|<\frac{1}{10^5}.\]}

\pr{2013 IMOSL A4}{设 $n$ 是正整数, 正整数 $a_1,a_2,\cdots,a_n$ 满足$a_1\leq a_2\leq\cdots\leq a_n\leq n+a_1$ , 且 $a_{a_i}\leq n+i-1,$ $1\leq i\leq n$(其中脚标按模 $n$ 理解). 求证:
\[a_1+a_2+\cdots+a_n\leq n^2.\]}

\pr{2012 IMOSL A2}{\begin{enumerate}
	\item 是否能将 $\mathbb{Z}$ 表示成三个非空子集 $A,B,C$ 的不交并, 使得 $A+B,B+C,C+A$ 两两不交?
	\item 是否能将 $\mathbb{Q}$ 表示成三个非空子集 $A,B,C$ 的不交并, 使得 $A+B,B+C,C+A$ 两两不交?
\end{enumerate}}

\pr{2011 IMO P1}{对于由四个不同的正整数组成的集合 $A=\{a_1,a_2,a_3,a_4\}$, 定义$s_A=a_1+a_2+a_3+a_4$. 设恰有 $n_A$ 对 $(i,j)$, $1\leq i<j\leq4$, 使得 $a_i+a_j\mid s_A$. 求所有的集合 $A$, 使得$n_A$ 达到最大值.}

\pr{2011 IMOSL A2}{求所有的正整数数列 $x_1,x_2,\cdots,x_{2011}$ , 使得对每个正整数 $n$, 都存在整数 $a$, 满足
\[x_1^n+2x_2^n+\cdots+2011x_{2011}^n=a^{n+1}+1.\]}

\pr{2011 IMOSL A5}{设 $n$ 是正整数. 求证:  可以将集合$\{2,3,\cdots,3n+1\}$划分为 $n$ 个三元子集的不交并, 使得每个子集中的三个数都能构成钝角三角形的三边长.}

\pr{2010 IMOSL A3}{设非负实数 $x_1,x_2,\cdots,x_{100}$ 满足: 对$1\leq i\leq100$, 有 $x_i+x_{i+1}+x_{i+2}\leq1$, 其 中$x_{101}=x_1$, $x_{102}=x_2$. 求$\sum_{i=1}^{100} x_ix_{i+2}$ 的最大值.}

\pr{2010 IMOSL A4}{设数列 $\{x_n\}$ 满足 $x_1=1,x_{2k}=-x_k$, $x_{2k-1}=(-1)^{k+1}x_k$, $k\geq1$. 求证: 对任意正整数 $n$, $x_1+x_2+\cdots+x_n\geq0.$}

\pr{2009 IMOSL A1}{已知 $2009$ 个非退化的三角形, 将每个三角形的三边分别染上蓝、红、白色. 对于每种颜色, 将边按长度排序. 设蓝色边的长度为 $b_1\leq b_2\leq\cdots\leq b_{2009}$ ; 红色边的长度为 $r_1\leq r_2\leq\cdots\leq r_{2009}$ ; 白色边的长度为$w_1\leq w_2\leq\cdots\leq w_{2009}$ . 求最大的整数 $k$, 使得存在$k$ 个下标 $j$, 满足以 $b_j,r_j,w_j$ 为边长能构成一个非退化的三角形.}

\pr{2006 IMOSL A3}{设数列 $c_0,c_1,c_2\cdots$ 满足: $c_0=1$, $c_1=0$, $c_{n+2}=c_{n+1}+c_n$, $n\geq0$. 考虑有序数对 $(x,y)$ 构成的集合 $S$, 其中 $(x,y)$ 满足: 有一个由正整数构成的有限集$J$, 使得
\[x=\sum_{j\in J}c_j,\quad y=\sum_{j\in J}c_{j-1}.\]
求证: 存在实数 $\alpha,\beta,m,M$, 使得对 $x,y\in\mathbb{N}$, $(x,y)\in S$ 的充分必要条件是 $m<\alpha x+\beta y<M$.\par
注: 空集的元素和为$0$.}

\pr{2003 IMOSL A1}{设实数 $a_{ij}$ 满足:当 $i=j$ 时, $a_{ij}$ 为正数; 当$i\neq j$ 时, $a_{ij}$ 为负数, 其中 $1\leq i,j\leq3$.求证: 存在正实数 $c_1,c_2,c_3$, 使得
\[a_{11}c_1+a_{12}c_2+a_{13}c_3,a_{21}c_1+a_{22}c_2+a_{23}c_3,a_{31}c_1+a_{32}c_2+a_{33}c_3\]
要么都是负数, 要么都是正数, 要么都是$0$.}

\pr{2003 IMOSL A3}{考虑两个正实数列 $a_1\geq a_2\geq a_3\geq\cdots,$ $b_1\geq b_2\geq b_3\geq\cdots$.记
\[A_n=a_1+a_2+\cdots+a_n,B_n=b_1+b_2+\cdots+b_n,\:n\geq1.\] 
设
\[c_i=\min\{a_i,b_i\},\quad C_n=c_1+c_2+\cdots+c_n,n\geq.1\]
\begin{enumerate}
\item 是否存在数列 $\{a_i\},\{b_i\}$, 使得数列 $\{A_n\},\{B_n\}$无界,而数列 $\{C_n\}$ 有界?
\item 若$b_i=\frac1i,i\geq1$,则(1)的结论是否改变?
\end{enumerate}}

\pr{2009 IMOSL A6}{设 $n$ 是正整数, $x_1,x_2,\cdots,x_n$, $y_1,y_2,\cdots,y_n$, $z_2,\cdots,z_{2n}$ 是正实数, 满足对 $1\leq i,j\leq n$, $z_{i+j}^2\geq x_iy_j$. 记 $M=\max\{z_2,z_3,\cdots,z_{2n}\}$, 求证:
\[\left(\frac{M+z_2+z_3+\cdots+z_{2n}}{2n}\right)^2\geq\left(\frac{x_1+x_2+\cdots+x_n}n\right)\left(\frac{y_1+y_2+\cdots+y_n}n\right).\]}

\section{国际数学奥林匹克预选题组合}

\pr{2022 IMOSL C1}{一个$\pm1$ - 序列是一个长为 $2022$ 的序列 $a_1,a_2,\cdots,a_{2022}$, 其中每个 $a_i\in\{-1,1\}$. 求最大的整数 $C$, 使得对任意一个$\pm1$ - 序列, 存在一个正整数 $k$ 和一列下标 $1\leq t_1<\cdots<t_k\leq2022$, 满足$t_{i+1}-t_i\leq2$ $(i=1, \cdots,k-1)$, 且
\[\left|\sum_{i=1}^ka_{t_i}\right|\geq C.\]}

\pr{2013 IMOSL C1}{给定正整数 $n$. 求最小的正整数 $k$, 使得对任意正整数 $d$ 及任意不超过 $1$ 的正实数 $a_1,a_2,\cdots,a_d$ , 只要$a_1+a_2+\cdots+a_d=n$ , 那么总能将这些数分成不超过 $k$ 组, 满足每组中的数之和不超过 $1$.}

\pr{2009 IMOSL C3}{设 $n$ 是正整数, 数列 $c_1,c_2,\cdots,c_{n-1}$ 的各项为 $0$ 或 $1$ . 数列 $a_0,a_1,\cdots,a_n$ 和 $b_0,b_1,\cdots,b_n$ 满足$a_0=b_0=1$, $a_1=b_1=7$, 且对于任意$1\leq i\leq n-1$, 
\[a_{i+1}=\begin{cases}2a_{i-1}+3a_i,&c_i=0;\\3a_{i-1}+a_i,&c_i=1,\end{cases},
\quad
b_{i+1}=\begin{cases}2b_{i-1}+3b_i,&c_{n-i}=0;\\3b_{i-1}+b_i,&c_{n-i}=1.\end{cases}\]
求证: $a_n=b_n$.}

\pr{2008 IMOSL C5}{设 $k,l$ 是正整数, 实数 $x_i\in[0,1],$ $1\leq i\leq k+l$.称集合 $S=\{x_1,x_2,\cdots,x_{k+l}\}$ 的 $k$元子集 $A$ 是“好的”, 如果
\[\left|\frac1k\sum_{x_i\in A}x_i-\frac1l\sum_{x_j\in S\setminus A}x_j\right|\leq\frac{k+l}{2kl}.\]
求证: 好子集至少有$\displaystyle\frac2{k+l}\binom{k+l}{k}$}

\pr{2007 IMOSL C1}{设整数 $n\geq2$. 求满足下列条件的所有数列 $a_1,a_2,$ $\cdots,a_{n^2+n}$:
\begin{enumerate}
	\item 对 $1\leq i\leq n^2+n$ , $a_i\in\{0,1\}$;
	\item 对 $0\leq i\leq n^2-n$ ,
	\[a_{i+1}+a_{i+2}+\cdots+a_{i+n}<a_{i+n+1}+a_{i+n+2}+\cdots+a_{i+2n}.\]
\end{enumerate}}

\pr{2007 IMOSL C4}{设 $A_0=(a_1,a_2,\cdots,a_n)$ 是实数数列. 对每个非负整数 $k$, 由数列 $A_k=(x_1,x_2,\cdots,x_n)$ 来构造一个新的数列$A_{k+1}$, 满足以下条件:
\begin{enumerate}
	\item 选取 $\{1,2,\cdots,n\}$ 的一个划分 $(I,J)$,使
	\[\left|\sum_{i\in I}x_i-\sum_{j\in J}x_j\right|\]
	取得最小值(允许 $I$ 或 $J$ 是空集, 这种情况的和为 $0$ ). 如果有多于一个这样的划分, 任选其中一个;
	\item 设数列 $A_{k+1}=(y_1,y_2,\cdots,y_n)$, 其中若 $i\in I$ , 
	则 $y_i=x_i+1$; 若 $i\in J$, 则 $y_i=x_i-1$.
\end{enumerate}
求证: 存在非负整数 $k$, 使得数列 $A_k$ 中包含一项 $x$, 满足 $|x|\geq n/2$.}

\section{欧洲女子数学奥林匹克}

\pr{2022 EGMO P4}{给定整数 $n\geq2.$ 求最大的正整数 $N$, 使得存在$N+1$ 个实数 $a_0,a_1,\cdots,a_N$, 满足
\begin{enumerate}
	\item $a_0+a_1=-1/n$;
	\item 对任意 $1\leq k\leq N-1$ ,
	\[(a_k+a_{k-1})(a_k+a_{k+1})=a_{k-1}-a_{k+1}.\]
\end{enumerate}}

\pr{2020 EGMO P2}{求所有的非负实数组 $(x_1,x_2,\cdots,x_{2020})$, 满足: 
\begin{enumerate}
	\item $x_1\leq x_2\leq\cdots\leq x_{2020}\leq x_1+1$ ;
	\item 存在 $x_1,x_2,\cdots,x_{2020}$ 的一个排列$y_1,y_2,\cdots,y_{2020}$ , 使得
	\[\sum_{i=1}^{2020}((x_i+1)(y_i+1))^2=8\sum_{i=1}^{2020}x_i^3.\]
\end{enumerate}}

\pr{2019 EGMO P5}{设整数 $n\geq2$, $a_1,a_2,\cdots,a_n$ 是正整数. 求证: 存在正整数 $b_1,b_2,\cdot,b_n$满足以下条件:
\begin{enumerate}
	\item 对任意 $1\leq i\leq n$, 均有 $a_i\leq b_i$ ;
	\item $b_1,b_2,\cdots,b_n$ 两两模 $n$ 不同余;
	\item \[b_1+b_2+\cdots+b_n\leq n\left (\frac{n-1}2+\fl{\frac{a_1+a_2+\cdots+a_n}n}\right).\]
\end{enumerate}}

\pr{2016 EGMO P1}{设 $n$ 是正奇数, $x_1,x_2,\cdots,x_n$ 是非负实数.求证:
\[\min_{1\leq i\leq n}\{x_i^2+x_{i+1}^2\}\leq\max_{1\leq i\leq n}\{2x_ix_{i+1}\},\]
其中 $x_{n+1}=x_1.$}

\pr{2014 EGMO P1}{求所有的实数 $t$, 使得当 $a,b,c$ 为某个三角形的三边长时, $a^2+bct,b^2+cat,c^2+abt$ 也为某个三角形的三边长.}

\pr{2012 EGMO P2}{给定正整数 $n$. 求最大的正整数 $m$, 使得存在一个 $m$ 
行 $n$ 列的实矩阵, 满足对于任意两个不同的行 $(a_1,a_2,\cdots,a_n)$  和 $(b_1,b_2,\cdots,b_n)$, 有
\[\max\{|a_1-b_1|,|a_2-b_2|,\cdots,|a_n-b_n|\}=1.\]}

\section{国际大都市数学奥林匹克}

\pr{2020 International Mathematical Tournament of Towns P3}{设整数 $n\geq2$. 铸币厂需要铸造一套有 $n$ 种面值的硬币, 每种面值都是正整数, 且每种面值的硬币个数不限. 称面值 $\{a_1,a_2,\cdots,a_n\}$ 是"幸运的", 如果 $a_1+a_2+\cdots+a_n$ 只能通过唯一的一种方式给出, 即每种面值的硬币各一个. 求证:
\begin{enumerate}
	\item 存在幸运的面值 $a_1,a_2,\cdots,a_n$, 使得
	\[a_1+a_2+\cdots+a_n<n\cdot2^n;\]
	\item 对任意幸运的面值 $a_1,a_2,\cdots,a_n$, 均有
	\[a_1+a_2+\cdots+a_n>n\cdot2^{n-1}.\]
\end{enumerate}}

\pr{2016 International Mathematical Tournament of Towns P3}{设 $A_1A_2\cdots A_n$ 是内接于 $\odot O$ 的凸 $n$ 边形, $O$ 是 $A_1A_2\cdots A_n$ 内的点. $B_1,B_2,\cdots,B_n$ 分别是边 $A_1A_2,A_2A_3,\cdots,A_nA_1$ 上的点 (均不与顶点重合). 求证:
\[\frac{B_1B_2}{A_1A_3}+\frac{B_2B_3}{A_2A_4}+\cdots+\frac{B_nB_1}{A_nA_2}>1.\]}

\section{捷克波兰斯洛伐克数学竞赛}

\pr{2023 Czech-Polish-Slovak Match P2}{设 $a_1,a_2,\cdots,a_n$ 是实数, 满足对任意 $1\leq k\leq n$, 均有
\[n\cdot a_k\geq\sum_{i=1}^ka_i^2.\]
求证: 存在至少 $n/10$ 个 $k$ 使 $a_k\leq1000$.}

\section{美国数学奥林匹克}

\pr{2012 USAMO P1}{求所有的整数 $n\geq3$, 使得对任意满足
\[\max\{a_1,a_2,\cdots,a_n\}\leq n\cdot\min\{a_1,a_2,\cdots,a_n\}\]
的 $n$ 个正实数 $a_1,a_2,\cdots,a_n$, 其中都存在三个为一个锐角三角形的三边长.}

\pr{2012 USAMO P6}{设整数 $n>2$,实数 $x_1,x_2,\cdots,x_n$ 满足
\[x_1+x_2+\cdots+x_n=0,\ x_1^2+x_2^2+\cdots+x_n^2=1.\]
对于 $\{1,2,\cdots,n\}$ 的子集 $A$,定义 $S_A=\sum_{i\in A}x_i$, 规定 $S_\emptyset=0.$\par
求证: 对任意正数 $\lambda$, 满足 $S_A\geq\lambda$ 的集合 $A$ 的个数
不超过 $2^{n-3}/\lambda^2$, 并求出等号成立时的 $x_1,x_2,\cdots,x_n,\lambda$.}

\pr{2010 USAMO P3}{	设 $a_1,a_2,\cdots,a_{2010}$是正实数, 满足对任 $1\leq i<j\leq2010$, 都有 $a_ia_j\leq i+j$. 求 $a_1a_2\cdots 	a_{2010}$ 的最大值. }

\pr{2006 USAMO P4}{求所有的正整数 $n$, 使得存在整数 $k\geq2$ 及正有理数$a_1,a_2,\cdots,a_k$ , 满足
\[a_1+a_2+\cdots+a_k=a_1a_2\cdots a_k=n.\] }

\pr{2000 USAMO P6}{设 $a_1,a_2,\cdots,a_n,b_1,b_2,\cdots,b_n$ 是非负实数, 求证: 
\[\sum_{i,j=1}^n\min\{a_ia_j,b_ib_j\}\leq\sum_{i,j=1}^n\min\{a_ib_j,a_jb_i\}.\]}

\pr{1997 USAMO P6}{设 $a_1,a_2,\cdots,a_{1997}$ 是非负整数, 满足对任意$1\leq i,j\leq1997$, 若$i+j\leq1997$, 则
\[a_i+a_j\leq a_{i+j}\leq a_i+a_j+1.\]
求证: 存在实数 $x$, 使得对任意 $1\leq n\leq1997$, 均有$a_n=\fl{nx}.$}

\section{美国TSTST}

\pr{2019 American TSTST P4}{求最小的实数 $\lambda$, 使得对任意不超过 $1$ 且和为 $50$ 的正实数 $x_1,x_2,\cdots,x_{100}$, 总存在集合 $\{1,2,\cdots,100\}$ 的划分 $(A,B)$, 满足 $|A|=|B|$, 且
\[\left|\sum_{i\in A}x_i-\sum_{j\in B}x_j\right|\leq\lambda.\]}

\pr{2015 American TSTST P1}{设正整数 $m<n,\quad a_1,a_2,\cdots,a_n$ 是实数. 对$1\leq k\leq n$, 称$k$ 是"好的", 如果存在 $1\leq l\leq m$, 使得 $a_k+a_{k+1}+\cdots+a_{k+l-1}\geq0$ , 其中脚标按模$n$ 理解. 用 $T$ 表示所有好数构成的集合, 求证: 
\[\sum_{k \in T}a_k\ge 0.\]}

\section{北大夏令营}
\pr{2023 北大夏令营 P2}{对正整数 $n$, 用 $S(n)$ 表示 $0\sim n-1$ 在十进制中的数码和之和. 求证: 对任意正整数 $m,n$,
\[S(m+n)\geq S(m)+S(n)+\min\{m,n\}.\]}

\pr{2023 北大夏令营 P5}{给定正整数 $n>m$.求所有的数组 $(i_1,i_2,\cdots,i_m)$ $(1\leq i_1<i_2<\cdots<i_m\leq n)$, 使得对任意满足 $\sum_{i=1}^nx_i=0$ 的实数组 $(x_1,x_2,\cdots,x_n)$ $\left(x_1<x_2<\cdots<x_n\right)$, 都有
\[\sum_{k=1}^{m}x_{i_k}>0.\]}

\pr{2022 北大夏令营 P2}{设数列 $\{f_n\}_{-\infty}^{+\infty}$ 满足只有 $f_1,f_2,\cdots,f_{2022}$ 可能不为 $0$ . 令
\[M(n)=\max_{r,s\geq0}\left.\left(\sum_{i=n-r}^{n+s}|f_i|\right)\right/(r+s+1).\]
求证:
\[\sum_{n=1}^{2023}\lvert M(n)-M(n-1)\rvert\leq\sum_{n=1}^{2023}\lvert f_n-f_{n-1}\rvert.\]}

\pr{2021 北大夏令营 P1}{设 $a_1,a_2,a_3,a_4,k$ 是两两不同的正整数, 且不小于 $80$ , 满足 $a_1^2+a_2^2+a_3^2+a_4^2-4k^2$ 是正整数. 求$
(a_1^2+a_2^2+a_3^2+a_4^2-4k^2)\cdot k^2$的最小值.}

\pr{2019 北大夏令营 P6}{对正实数 $a_1,a_2,\cdots,a_n$ , 定义
\[\sigma(a_1,a_2,\cdots,a_n)=\min\left\{\left|\sum_{k=1}^ne_ka_k\right|:e_k=\pm1\right\}.\]
求最小的实数 $\lambda$, 使得
\[\sigma(a_1,a_2,\cdots,a_n)\left(\sum_{k=1}^na_k\right)\leq\lambda\sum_{k=1}^na_k^2\]
对任意正实数 $a_1,a_2,\cdots,a_n$ 都成立.}

\pr{2018 北大夏令营 P3}{设实数 $a_1,a_2,\cdots,a_{2018}$ 满足 $|a_{i+1}-a_i|\leq1$ $(1\leq i\leq2018)$, 其中 $a_{2019}=a_1$. 求
\[\sum_{i=1}^{2018}\lvert a_i\rvert-\left|\sum_{i=1}^{2018}a_i\right|\]
的最大值.}

\pr{2018 北大夏令营 P5}{
设实数 $a>2$, 整数 $0\leq j\leq n$. 求证:
\[\sum_{k=0}^n(-1)^{k-j}a^{-k^2+2kj}>0.\]}

\section{北大金秋营}

\pr{2020 北大金秋营 P1}{给定正整数 $n$. 设非负实数 $a_1,a_2,\cdots,a_n$ 的和为 $1$, 记 $S$ 为如下 $2^n$ 个实数

\[\varepsilon_1a_1+\varepsilon_2a_2+\cdots+\varepsilon_na_n,\varepsilon_i\in\{-1,1\}\]

中所有正数之和. 求 $S$ 的最小可能值.}

\pr{2019 北大金秋营 P4}{设 $\theta_1,\theta_2,\cdots,\theta_l$ 是实数. 求证: 存在正整数 $k$ 和正实数 $a_1,a_2,\cdots,a_k$, 满足 $\sum_{i=1}^{k}a_i=1$, 且对任意正整数$n\leq k$ 和 $m\leq l$, 都有
\[\left|\sum_{j=1}^{n}a_j\sin{(j\theta_m)}\right|\le\frac{1}{2018n}.\]}

\pr{2017 北大金秋营 P4}{求最小的实数 $\lambda$, 使得对一切满足 $a_i<2^i$ 的正实数$a_1,a_2,\cdots,a_n$ , 都有
\[\sum_{1\le i,j\le n}\{a_ia_j\}\leq\lambda\sum_{i=1}^n\{a_i\}.\]}

\pr{2016 北大金秋营 P2}{设 $a_1,a_2,\cdots,a_n,b_1,b_2,\cdots,b_n$ 是 $1,2,\cdots,2n$ 的一个排列, 求
\[\sum_{i=1}^n |a_ib_i-a_{i+1}b_{i+1}|\]
的最小值.}




\end{document}
