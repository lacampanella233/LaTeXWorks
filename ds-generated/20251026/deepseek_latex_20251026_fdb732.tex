\documentclass[tikz,border=5pt]{standalone}
\usetikzlibrary{arrows.meta, decorations.markings}

\begin{document}
\begin{tikzpicture}[
    vertex/.style={circle, draw, fill=white, inner sep=1.5pt, minimum size=0.6cm},
    edge/.style={
        semithick,
        postaction={
            decorate,
            decoration={
                markings,
                mark=at position 0.5 with {\arrow{Stealth[scale=0.7]}}
            }
        }
    }
]

% 放置顶点
\foreach \i in {1,...,30} {
    \node[vertex] (v\i) at (90-12*\i:8cm) {\i};
}

% 绘制边,按终点的gcd染色
\foreach \x in {1,...,30} {
    \pgfmathsetmacro{\d}{gcd(\x,30)}
    \pgfmathsetmacro{\modulus}{30/\d}
    \foreach \y in {1,...,30} {
        \ifnum \y=\x
            % skip self
        \else
            \pgfmathtruncatemacro{\remainder}{mod(\y-1, \modulus)}
            \ifnum \remainder=0
                \pgfmathtruncatemacro{\ygcd}{gcd(\y,30)}
                \pgfmathsetmacro{\bendamt}{8}
                \pgfmathsetmacro{\benddir}{rnd>0.5 ? "left" : "right"}
                
                % 根据ygcd值选择颜色和线型
                \ifnum\ygcd=1
                    \def\edgecolor{red!70!black}
                    \def\edgestyle{solid}
                \fi
                \ifnum\ygcd=2
                    \def\edgecolor{blue!70!black}
                    \def\edgestyle{solid}
                \fi
                \ifnum\ygcd=3
                    \def\edgecolor{green!70!black}
                    \def\edgestyle{solid}
                \fi
                \ifnum\ygcd=5
                    \def\edgecolor{orange!70!black}
                    \def\edgestyle{solid}
                \fi
                \ifnum\ygcd=6
                    \def\edgecolor{purple!70!black}
                    \def\edgestyle{solid}
                \fi
                \ifnum\ygcd=10
                    \def\edgecolor{brown!70!black}
                    \def\edgestyle{solid}
                \fi
                \ifnum\ygcd=15
                    \def\edgecolor{cyan!70!black}
                    \def\edgestyle{solid}
                \fi
                \ifnum\ygcd=30
                    \def\edgecolor{coral!70!black}
                    \def\edgestyle{solid}
                \fi
                
                \if\benddir left
                    \draw[edge, \edgestyle, draw=\edgecolor] 
                        (v\x) to [bend left=\bendamt] (v\y);
                \else
                    \draw[edge, \edgestyle, draw=\edgecolor] 
                        (v\x) to [bend right=\bendamt] (v\y);
                \fi
            \fi
        \fi
    }
}

% 添加图例
\begin{scope}[shift={(-2,-12)}]
    \node[anchor=north west] at (0,0) {\textbf{Sample (according to terminal gcd)}};
    
    % 可能的gcd值: 1,2,3,5,6,10,15,30
    \foreach \gcdvalue/\ypos in {30/0.5, 15/1.0, 10/1.5, 6/2.0, 5/2.5, 3/3.0, 2/3.5, 1/4.0} {
        \ifnum\gcdvalue=1
            \def\legendcolor{red!70!black}
            \def\legendstyle{solid}
        \fi
        \ifnum\gcdvalue=2
            \def\legendcolor{blue!70!black}
            \def\legendstyle{solid}
        \fi
        \ifnum\gcdvalue=3
            \def\legendcolor{green!70!black}
            \def\legendstyle{solid}
        \fi
        \ifnum\gcdvalue=5
            \def\legendcolor{orange!70!black}
            \def\legendstyle{solid}
        \fi
        \ifnum\gcdvalue=6
            \def\legendcolor{purple!70!black}
            \def\legendstyle{solid}
        \fi
        \ifnum\gcdvalue=10
            \def\legendcolor{brown!70!black}
            \def\legendstyle{solid}
        \fi
        \ifnum\gcdvalue=15
            \def\legendcolor{cyan!70!black}
            \def\legendstyle{solid}
        \fi
        \ifnum\gcdvalue=30
            \def\legendcolor{coral!70!black}
            \def\legendstyle{solid}
        \fi
        
        \draw[\legendstyle, draw=\legendcolor, semithick] 
            (0,-\ypos) -- (0.5,-\ypos) 
            node[right, black] {gcd = \gcdvalue};
    }
\end{scope}

\end{tikzpicture}
\end{document}