\documentclass{article}
\usepackage[UTF8]{ctex}
\usepackage{amsmath,mathtools,geometry,enumerate,amsfonts,amssymb}
\geometry{scale=0.8}

\title{L-Function}
\author{}
\date{}

\begin{document}
\maketitle
Define
\[L(X,Y,t)\]
is the the level of love of $X$ for $Y$ where
\[{X,Y}\in \{x|x\text{ is a person}\};\quad t\in\mathbb{R}_+;\quad L(X,Y,t)\in\mathbb{R}_+.\]
%For example, in Xi'an Tieyi Middle School, 
%\[L(\text{WYD},\text{XLF},t)=L(\text{XLF},\text{WYD},t)\to+\infty, \text{when }t\to+\infty.\]\par
And we can find another good example for explaning this function:
\paragraph{Example (of a (fake) partial differential equation): }\textit{Assume A and B are a boy and a girl. If A's love for B is two times the velocity of the increases or the decreases(at this time the velocity is negative) of B's love for A, and B's love for A is $1/2$ times the velocity of the increases or decreases(such as above) of A's love for B. And we also know that
\[L(A,B,0)=0,\quad\left.\frac{\partial L(A,B,0)}{\partial t}\right|_{t=0}=1.\]
If
\[L(A,B,t)\ge 520\quad and\quad L(B,A,t)\ge 520,\]
then they will get married.}\par
\textit{Find the value of $t$ when they get married at the first time.}\par
the answer is 
\[t=\mathrm{arcsinh}\hspace{0.3em}1040.\]
\end{document}