\documentclass{article}
\usepackage[UTF8]{ctex}
\usepackage{amsmath,mathtools,geometry}
\title{\vspace*{-2.5cm}\textbf{\songti{\LARGE 三角函数公式}} \vspace*{-2cm}}
\geometry{a5paper,scale=0.8}
\date{}
\author{}

\begin{document}
\vspace*{\fill}
\begin{center}
	{\textbf{{\huge\hspace*{\fill+2cm}三\hspace{\fill}角\hspace{\fill}函\hspace{\fill}数\hspace{\fill}公\hspace{\fill}式\hspace*{\fill+2cm}}}}
\end{center}
\pagestyle{empty}
\vspace*{\fill+3cm}
\newpage
\quad
\newpage
\maketitle
\pagestyle{plain}
\setcounter{page}{1}
\section{{\large 同角三角函数的关系}}
\textbf{1.}平方关系:
\[\sin^2\alpha+\cos^2\alpha=1,\tan^2\alpha+1=\sec^2,\cot^2\alpha+1=\csc^2\alpha.\]
\par\textbf{2.}倒数关系:
\[\sin\alpha\cdot\csc\alpha=1,\cos\alpha\cdot\sec\alpha=1,\tan\alpha\cdot\cot\alpha=1.\]
\par\textbf{3.}商数关系:
\[\frac{\sin\alpha}{\cos\alpha}=\tan\alpha,\frac{\cos\alpha}{\sin\alpha}=\cot\alpha.\]

\section{{\large 诱导公式}}
\textbf{1.}
\[\sin(k\pi+\alpha)=\pm\sin\alpha;\quad
\cos(k\pi+\alpha)=\pm\cos\alpha;\]
\[\tan(k\pi+\alpha)=\pm\tan\alpha;\quad
\cot(k\pi+\alpha)=\pm\cot\alpha.\]
\par\textbf{2.}
\[\sin\left(\left(k+\frac12\right)\pi+\alpha\right)=\pm\cos\alpha;\quad\cos\left(\left(k+\frac12\right)\pi+\alpha\right)=\pm\sin\alpha;\]
\[\tan\left(\left(k+\frac12\right)\pi+\alpha\right)=\pm\cot\alpha;\quad\cot\left(\left(k+\frac12\right)\pi+\alpha\right)=\pm\tan\alpha;\]
口诀:{\kaishu 奇变偶不变,符号看象限.}

\section{{\large 和差倍角公式}}
\textbf{1.}
\[\sin(\alpha\pm\beta)=\sin\alpha\cos\beta\pm\sin\beta\cos\alpha;\quad\cos(\alpha\pm\beta)=\cos\alpha\cos\beta\mp\sin\alpha\sin\beta;\]
\[\tan(\alpha\pm\beta)=\frac{\tan\alpha\pm\tan\beta}{1\mp\tan\alpha\tan\beta}.\]
\par\textbf{2.}
\[\sin2\alpha=2\sin\alpha\cos\alpha;\quad \cos2\alpha=\cos^2\alpha-\sin^2\alpha=2\cos^2\alpha-1=1-2\sin^2\alpha;\]
\[\tan2\alpha=\frac{2\tan\alpha}{1+\tan^2\alpha}.\]
\section{{\large 和(差)积互化公式}}

\textbf{1.}和差化积:
\[\sin\alpha+\sin\beta=2\sin\frac{\alpha+\beta}{2}\sin\frac{\alpha-\beta}{2};\quad\sin\alpha-\sin\beta=2\cos\frac{\alpha+\beta}{2}\sin\frac{\alpha-\beta}{2};\]
\[\cos\alpha+\cos\beta=2\cos\frac{\alpha+\beta}{2}\cos\frac{\alpha-\beta}{2};\quad\cos\alpha-\cos\beta=-2\sin\frac{\alpha+\beta}{2}\sin\frac{\alpha-\beta}{2}.\]
\par\textbf{2.}积化和差:
\[\sin\alpha\sin\beta=-\frac{1}{2}[\cos(\alpha+\beta)-\cos(\alpha-\beta)];\quad\cos\alpha\cos\beta=\frac{1}{2}[\cos(\alpha+\beta)+\cos(\alpha-\beta)].\]
\[\sin\alpha\cos\beta=\frac{1}{2}[\sin(\alpha+\beta)+\sin(\alpha-\beta)];\quad\cos\alpha\sin\beta=\frac{1}{2}[\sin(\alpha+\beta)-\sin(\alpha-\beta)].\]
\section{{\large 万能置换公式}}\vspace{-0.75cm}

\[\sin\alpha=\frac{2\tan\dfrac{\alpha}{2}}{1+\tan^2\dfrac{\alpha}{2}};\quad\cos\alpha=\frac{1-\tan^2\dfrac{\alpha}{2}}{1+\tan^2\dfrac{\alpha}{2}};\quad\tan\alpha=\frac{2\tan\dfrac{\alpha}{2}}{1-\tan^2\dfrac{\alpha}{2}}.\]
\section{{\large 三角形中的公式}}
\textbf{1.}正弦定理,余弦定理,射影定理.
\par\textbf{2.}
\[\sum\limits_{cyc}\sin A=4\prod\limits_{cyc} \cos\frac{A}{2};\]
\[\sum\limits_{cyc}\cos A=1+4\prod\limits_{cyc} \sin\frac{A}{2};\]
\[\sum\limits_{cyc}\tan A=\prod\limits_{cyc}\tan A;\]
\[\sum\limits_{cyc}\sin^2A=2+2\prod\limits_{cyc}\cos A;\]
\[\sum\limits_{cyc}\left(\tan\frac{A}{2}\tan\frac{B}{2}\right)=1.\]
\section{{\large 三角形中的不等式}}
\[\prod\limits_{cyc}\cos A\le\frac{1}{8};\]
\[1<\sum\limits_{cyc}\cos A\le\frac{3}{2};\]
\[\sum\limits_{cyc}\sin A\le\frac{3\sqrt{3}}{2}.\]

\section{{\large 其他}}
\textbf{1.}\[(\sin\alpha\pm\cos\alpha)^2=1\pm\sin2\alpha.\]
\par\textbf{2.}
\[\frac{1+\tan\alpha}{1-\tan\alpha}=\frac{\sin\alpha+\cos\alpha}{\cos\alpha-\sin\alpha}=\tan\left(\alpha+\frac{\pi}{4}\right).(\text{定义域略})\]
\par\textbf{3.}
\[\tan\alpha+\cot\alpha=\frac{2}{\sin2\alpha};\]
\[\tan\alpha-\cot\alpha=-2\cot\alpha.(\text{定义域略})\]
\par\textbf{4.}
\[\sin(\alpha+\beta)\sin(\alpha-\beta)=\sin^2\alpha-\sin^2\beta=\cos^2\beta-\cos^2\alpha;\]
\[\cos(\alpha+\beta)\cos(\alpha-\beta)=\cos^2\alpha-\sin^2\beta.\]
\par\textbf{5.}
\[\sin3\alpha=4\sin(60^\circ-\alpha)\sin\alpha\sin(60^\circ+\alpha);\]
\[\cos3\alpha=4\cos(60^\circ-\alpha)\cos\alpha\cos(60^\circ+\alpha);\]
\[\tan3\alpha=\tan(60^\circ-\alpha)\tan\alpha\tan(60^\circ+\alpha);\]
\newpage
\pagestyle{empty}
\quad
\newpage
\quad
\end{document}