\documentclass[UTF8]{ctexart}
\usepackage{amsmath}
\usepackage{mathtools}
\usepackage{cases}
\title{一元三次方程的求根公式及其证明}
\begin{document}
\maketitle
\begin{flushleft}
已知:关于$x$的三次方程$x^{3}+bx^{2}+cx+d=0$,求此方程的精确解.\\
解:令$x=t-\frac{b}{3}$,则有\\
\end{flushleft}
\begin{align*}
&x^{3}+bx^{2}+cx+d\\
=&(t-\frac{b}{3})^{3}+b(t-\frac{b}{3})^{2}+c(t-\frac{b}{3})+d\\
=&t^{3}-bt^{2}+\frac{1}{3}tb^{2}-\frac{b^{3}}{27}+bt^{2}-\frac{2}{3}b^{2}t+\frac{b^{3}}{9}+ct-\frac{bc}{3}+d\\
=&t^{3}+(c-\frac{b^{2}}{3})t+(\frac{2}{27}b^{3}-\frac{bc}{3}+d)\\
=&0
\end{align*}
令$p=c-\frac{b^{2}}{3},q=\frac{2}{27}b^{3}-\frac{bc}{3}+d$,则原方程化为
\[t^{3}+pt+q=0\]
下面求解这个方程.\\
已知公式:
\[(t+y+z)(t+\omega y+\omega ^{2}z)(t+\omega ^{2}y+\omega z)=t^{3}+y^{3}+z^{3}-3tyz\]
其中$\omega =\frac{-1+\sqrt{3}\mathrm{i}}{2}$.
令
\begin{numcases}{}
-3yz=p\\
y^{3}+z^{3}=q
\end{numcases}则有
\[t^{3}+pt+q=(t+y+z)(t+\omega y+\omega ^{2}z)(t+\omega ^{2}y+\omega z)=0\]
则\\\[
\begin{cases}
t_{1}=-y-z\\
t_{2}=-\omega y-\omega ^{2}z\\
t_{3}=-\omega ^{2}y-\omega z\\
\end{cases}\]
由$(1)$得
\[z=-\frac{p}{3y}\]
代入$(2)$,得
\[y^{3}+(-\frac{p}{3y})^{3}=q\]
\[y^{6}-qy^{3}-\frac{p^{3}}{27}=0\]
得
\[y=\sqrt[3]{\frac{q\pm\sqrt{q^{2}+\frac{4p^{3}}{27}}}{2}}\]
代入$(1)$,得
\[z=-\frac{p}{3\sqrt[3]{\frac{q\pm\sqrt{q^{2}+\frac{4p^{3}}{27}}}{2}}}=\sqrt[3]{\frac{q\mp\sqrt{q^{2}+\frac{4p^{3}}{27}}}{2}}\]
所以,
\[\begin{cases}
t_{1}=-\sqrt[3]{\frac{q\pm\sqrt{q^{2}+\frac{4p^{3}}{27}}}{2}}-\sqrt[3]{\frac{q\mp\sqrt{q^{2}+\frac{4p^{3}}{27}}}{2}}\\
t_{2}=-\omega \sqrt[3]{\frac{q\pm\sqrt{q^{2}+\frac{4p^{3}}{27}}}{2}}-\omega ^{2}\sqrt[3]{\frac{q\mp\sqrt{q^{2}+\frac{4p^{3}}{27}}}{2}}\\
t_{3}=-\omega ^{2}\sqrt[3]{\frac{q\pm\sqrt{q^{2}+\frac{4p^{3}}{27}}}{2}}-\omega \sqrt[3]{\frac{q\mp\sqrt{q^{2}+\frac{4p^{3}}{27}}}{2}}\\
\end{cases}\]
即
\[\begin{cases}
x_{1}=-\sqrt[3]{\frac{q\pm\sqrt{q^{2}+\frac{4p^{3}}{27}}}{2}}-\sqrt[3]{\frac{q\mp\sqrt{q^{2}+\frac{4p^{3}}{27}}}{2}}-\frac{b}{3}\\
x_{2}=-\omega \sqrt[3]{\frac{q\pm\sqrt{q^{2}+\frac{4p^{3}}{27}}}{2}}-\omega ^{2}\sqrt[3]{\frac{q\mp\sqrt{q^{2}+\frac{4p^{3}}{27}}}{2}}-\frac{b}{3}\\
x_{3}=-\omega ^{2}\sqrt[3]{\frac{q\pm\sqrt{q^{2}+\frac{4p^{3}}{27}}}{2}}-\omega \sqrt[3]{\frac{q\mp\sqrt{q^{2}+\frac{4p^{3}}{27}}}{2}}-\frac{b}{3}\\
\end{cases}\]
若用$b,c,d$代替$p,q$,则会得到以下复杂代数式:
\[\begin{cases}
x_{1}=-\sqrt[3]{\frac{(y^{3}+z^{3})\pm\sqrt{(y^{3}+z^{3})^{2}+\frac{4(-3yz)^{3}}{27}}}{2}}-\sqrt[3]{\frac{(y^{3}+z^{3})\mp\sqrt{(y^{3}+z^{3})^{2}+\frac{4(-3yz)^{3}}{27}}}{2}}-\frac{b}{3}\\
x_{2}=-\omega \sqrt[3]{\frac{(y^{3}+z^{3})\pm\sqrt{(y^{3}+z^{3})^{2}+\frac{4(-3yz)^{3}}{27}}}{2}}-\omega ^{2}\sqrt[3]{\frac{(y^{3}+z^{3})\mp\sqrt{(y^{3}+z^{3})^{2}+\frac{4(-3yz)^{3}}{27}}}{2}}-\frac{b}{3}\\
x_{3}=-\omega ^{2}\sqrt[3]{\frac{(y^{3}+z^{3})\pm\sqrt{(y^{3}+z^{3})^{2}+\frac{4(-3yz)^{3}}{27}}}{2}}-\omega \sqrt[3]{\frac{(y^{3}+z^{3})\mp\sqrt{(y^{3}+z^{3})^{2}+\frac{4(-3yz)^{3}}{27}}}{2}}-\frac{b}{3}\\
\end{cases}\]
\end{document}