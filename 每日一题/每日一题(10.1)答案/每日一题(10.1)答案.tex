\documentclass{article}
\usepackage[UTF8]{ctex}
\usepackage{amsmath,mathtools,geometry,fancyhdr,amssymb}
\geometry{a4paper,scale=0.7}
\pagenumbering{arabic}
\pagestyle{fancy}
\fancyfoot[C]{\thepage}
\fancyhead[L]{励志AQ班}
\fancyhead[R]{每日一题(10.1)答案}

\title{每日一题(10.1)答案}
\author{\kaishu 选题: 门宇翎\\\kaishu 答案制作: 程昊一}

\begin{document}
\maketitle
{\kaishu \textbf{1. }若实数$x$, $y$, $z$满足
\[x+y=4, |z+1|=xy+2y-9,\]
求$x+2y+3z$的值.\\\par}
\textbf{分析}\quad 我们还是选择配方.\par
\textbf{解}\quad 将(1)式代入(2)式, 得
\[|z+1|=x(4-x)+2(4-x)-9,\]
即
\[|z+1|+\left(x^2-2x+1\right)=0,\]
即
\[|z+1|+(x-1)^2=0.\]\par
所以, $z=-1, x=1, $代入(1)式, 得$y=3$.\\\par

{\kaishu \textbf{2. }若$n$为正整数, 证明: $n^2+n+1$不是完全平方数.}\\\par
\textbf{解}\quad 在这里, 我们通过证明$n^2+n+1$夹在两个完全平方数之间, 来证明它不是完全平方数. \par
$\because n^2<n^2+n+1<n^2+2n+1=(n+1)^2$, $\therefore n^2+n+1$不是完全平方数. 
\end{document}