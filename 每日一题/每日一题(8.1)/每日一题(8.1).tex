\documentclass{article}
\usepackage[UTF8]{ctex}
\usepackage{amsmath,mathtools,geometry,enumitem}
\geometry{a4paper,scale=0.7}
\title{每日一题(8.1)}
\author{门宇翎、李东宸}

\begin{document}
\maketitle
\textbf{1.}阅读材料:
{\kaishu
	\par 对于形如$\sqrt{m\pm\sqrt{n}}$的复合二次根式, 我们可以采取以下的方式化简: \\
	(1)\quad 找到合适的$a$和$b$, 使得$a+b=m$, $4ab=n$.\\
	(2)\quad 将原式做变形:
	\begin{align*}
		\sqrt{m\pm\sqrt{n}}&=\sqrt{a+b\pm\sqrt{4ab}}\\
		&=\sqrt{\left(\sqrt{a}\right)^2+\left(\sqrt{b}\right)^2\pm2\cdot\sqrt{a}\cdot\sqrt{b}}\\
		&=\sqrt{\left(\sqrt{a}\pm\sqrt{b}\right)^2}\\
		&=\left|\sqrt{a}\pm\sqrt{b}\right|.
	\end{align*}
	(3)\quad 即得答案:$\sqrt{m\pm\sqrt{n}}=\left|\sqrt{a}\pm\sqrt{b}\right|$.
}
\par 化简: 
\begin{itemize}
	\item[(1)]$\sqrt{5+2\sqrt{6}}$ ; 
	\item[(2)]$\sqrt{7-2\sqrt{12}}$ . 
\end{itemize}
{\kaishu (李东宸供题)}\\
\par \textbf{2.}证明: 若$a$, $b$是大于1的正整数, 则 $a^4+4b^4$是合数.\\
{\kaishu (门宇翎供题)} 

\end{document}