\documentclass{article}
\usepackage[UTF8]{ctex}
\usepackage{amsmath,mathtools,geometry,fancyhdr,amsfonts,amssymb}
\geometry{a4paper,scale=0.7}
\pagenumbering{arabic}
\pagestyle{fancy}
\fancyhead[L]{励志AQ班}
\fancyhead[R]{每日一题(9.1)答案}
\fancyfoot[C]{\thepage}
\title{每日一题(9.1)答案}
\author{\kaishu 选题: 李政毅\\\kaishu 答案制作:程昊一}

\begin{document}
\maketitle
\textbf{1. }
{\kaishu 已知:
\[\left(a^2+b^2+c^2\right)\left(x^2+y^2+z^2\right)=(ax+by+cz)^2\quad(a,b,c\neq0),\]
求证: $\dfrac{x}{a}=\dfrac{y}{b}=\dfrac{z}{c}$.\\\par}

\textbf{解}\quad 将原式展开, 得到
\[a^2x^2+b^2y^2+c^2z^2+a^2y^2+a^2z^2+b^2x^2+b^2z^2+c^2x^2+c^2y^2=a^2x^2+b^2y^2+c^2z^2+2abxy+2acxz+2bcyz,\]\par
很容易发现其中$a^2x^2+b^2y^2+c^2z^2$可以被抵消, 其余的部分可以分别分成三组并配方. 于是, 化简得
\[\left(a^2x^2+b^2y^2-2abxy\right)+\left(a^2x^2+c^2z^2-2acxz\right)+\left(b^2y^2+c^2z^2-2bcyz\right)=0,\]
即
\[(ax-by)^2+(ax-cz)^2+(by-cz)^2=0\]
所以
\[ax=by=cz\]
所以原命题成立.\\\par
\textbf{注}\quad 这道题的背景其实是\textbf{柯西不等式}, 完整形式如下:\par
\textbf{柯西不等式.} 设$a_1, a_2, \cdots, a_n, b_1, b_2, \cdots, b_n\in\mathbb{R}$, 则以下不等式成立: 
\[\left(a_1^2+a_2^2+\cdots+a_n^2\right)\left(b_1^2+b_2^2+\cdots+b_n^2\right)\ge(a_1b_1+\cdots+a_nb_n)^2,\]
即
\[\left(\sum\limits_{i=1}^na_i^2\right)\left(\sum\limits_{i=1}^nb_i^2\right)\ge\left(\sum\limits_{i=1}^na_ib_i\right)^2, \]
等号当且仅当存在实数$k$使得$a_1=kb_1, a_2=kb_2,\cdots, a_n=kb_n$时取等.\par
事实上, 此题就是$n=3$时取等的情况.\\\par
\textbf{2. }
{\kaishu 已知$a-b=4$, $ab+c^2+4=0$, 求$a+b$的值.}\\\par
\textbf{解(方法一)}\quad 将(1)式代入(2)式, 得
\[a(a-4)+4+c^2=0,\]
即
\[(a-2)^2+c^2=0.\]\par
所以$a=2$. 代入(1)式, 得$c=-2$. 所以, $a+b=0$.\par
\textbf{方法二}\quad 将原式变形, 得
\[\begin{cases}
	a-b=4\\a+b=-4-c^2
\end{cases},\]
则
\begin{align*}
	(a+b)^2&=(a-b)^2+4ab\\
	&=16+4(-4-c^2)\\
	&=-4c^2
\end{align*}
即
\[(a+b)^2+4c^2=0.\]\par
所以$a+b=0$.\par
\textbf{方法三}\quad (1)式平方, (2)式乘4, 得
\[\begin{cases}
	(a-b)^2=16\\4ab+4c^2+16=0
\end{cases},\]
上式代入下式, 得
\[(a-b)^2+4ab+4c^2=0,\]
即
\[(a+b)^2+4c^2=0,\]
所以$a+b=0$.
\end{document}