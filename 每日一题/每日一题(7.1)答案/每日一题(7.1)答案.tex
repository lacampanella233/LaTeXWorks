\documentclass{article}
\usepackage[UTF8]{ctex}
\usepackage{amsmath,mathtools,enumitem,geometry,fancyhdr}
\pagenumbering{arabic}
\pagestyle{fancy}
\fancyhead[L]{励志AQ班}
\fancyhead[R]{每日一题(7.1)答案}
\fancyfoot[C]{\thepage}
\renewcommand{\footrulewidth}{0.5pt}
\geometry{a4paper,scale=0.7}

\title{每日一题(7.1)答案}
\author{选题:程昊一、李衡岳\\答案制作:程昊一}
\begin{document}
\maketitle

\textbf{1.}{\CJKfamily{kai}
	证明:若$a,b>0,a,b\neq 1$,则
	\begin{itemize}
		\item[(1)]$\log_ax+\log_ay=\log_axy$\quad$(x,y>0)$;
		\item[(2)]$\log_ax^b=b\log_ax$\quad$(x>0)$;
		\item[(3)]$\log_ax=\dfrac{\log_bx}{\log_ba}$\quad$(x>0)$;
		\item[(4)]$\log_{a^x}b^y=\dfrac{y}{x}\log_ab$\quad$(x\neq 0)$.
	\end{itemize}
	(程昊一供题)
}
\par\textbf{分析}\quad 事实上,对数是指数的逆运算.如果我们对对数不熟悉,可以把对数转换为指数处理.
\par\textbf{解}\quad\textbf{(1).}设$\log_ax=m,\log_ay=n$,则根据指数的定义,我们有
\[\begin{cases}
a^m=x,\\a^n=y.
\end{cases}\]
两式相乘,有
\[a^{m+n}=xy,\]
即\[m+n=\log_axy,\]
所以\[\log_ax+\log_ay=\log_axy.\]
\par\textbf{(2).}设$\log_ax=m$,则
\[a^m=x\]
等式两边同时$b$次方,得
\[a^bm=x^,b\]
所以
\[bm=\log_ax^b,\]
即
\[\log_ax^b=b\log_ax.\]
\par\textbf{(3).}设$\log_ax=m$,则
\[a^m=x\]
等式两边同时以$b$为底进行底数运算,得到
\[m\log_ba=\log_bx,\]
即
\[m=\frac{\log_bx}{\log_ba},\]
所以
\[\log_ax=\dfrac{\log_bx}{\log_ba}.\]
这个公式被称为对数的\textbf{换底公式}.
\par\textbf{(4).}任取一个正数$c$,使得$c\neq 1$,那么
\begin{align*}
\log_{a^x}b^y&=\frac{\log_cb^y}{\log_ca^x}(\text{换底公式})\\
&=\frac{y\log_cb}{x\log_ca}(\text{公式\textbf{(2)}})\\
&=\frac{y}{x}\cdot\frac{\log_cb}{\log_ca}\\
&=\frac{y}{x}\log_ab(\text{换底公式的逆用}).
\end{align*}
\par 至此,我们将4个命题证明完毕.\\

\par\textbf{2.}{\CJKfamily{kai}
	有$n$个人,每个人的生日是完全随机且互不相关的.当$n$不小于多少时,存在两个生日相同的人的概率不小于$\dfrac{1}{2}$(假设一年有 365天)?\\
	(李衡岳供题)
}
\par\textbf{解}\quad %22:0.4757; 23:0.5073
我们从反面考虑,即考虑任意两个人的生日都不重复的概率.
\par 我们先\textbf{只}考虑第一个人.他(她)的生日是任意的,且没有其他的人,所以他(她)与其他人(事实上此时没有其他人)生日不重合的概率为1.
\par 我们再考虑第二个人.他(她)的生日是任意的,如果他(她)的生日为除去第一个人的生日的余下的364天,那么两个人的生日就不会重合,概率为$\dfrac{364}{365}$.
\par 我们再考虑第三个人,他(她)的生日也是任意的,且我们不希望与前两个人重合,概率(这三个人的生日均不重合的概率)为$\dfrac{364}{365} \times\dfrac{363}{365}$.
\par 同理,前4个人的生日互不重合的概率为$\dfrac{364}{365}\times \dfrac{363}{365}\times\dfrac{362}{365}$.
\par 那么,这$n$个人生日互不重合的概率为$\dfrac{364}{365}\times \dfrac{363}{365}\times\cdots\times\dfrac{366-n}{365}$,这$n$个人中存在两个生日相同的人的概率为
\begin{equation}\label{equ:gailv}
1-\dfrac{364}{365}\times\dfrac{363}{365}\times\cdots\times\dfrac{366-n}{365}.	
\end{equation}
\par 我们目前没有找到估算(\ref{equ:gailv})式的简单的初等方法.我们利用计算器得知,$n=23$时,这个式子的值约为0.4757;当$n=24$时,这个式子的值约为0.5073.当$n$增大时,这个式子的值也会越来越大.
\par 综上:当$n$不小于24时,存在两个生日相同的人的概率不小于$\dfrac{1}{2}$.
\end{document}