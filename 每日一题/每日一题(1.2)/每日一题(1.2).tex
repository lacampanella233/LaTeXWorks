\documentclass[UTF8]{ctexart}
\usepackage{amsmath}
\usepackage{mathtools}
\usepackage{geometry}
\usepackage{tikz}
\geometry{a4paper,scale=0.8}
\linespread{1.75}
\title{每日一题(1.2)}
\author{程昊一、李衡岳}
\begin{document}
\maketitle
\begin{flushleft}
1.如图,求$\angle A+\angle B+\angle C+\angle D+\angle E+\angle F+\angle G$的值.\\
2.用4个大小相等的方格可以组成7种图案(不计旋转),如图所示.证明:不能用他们密铺一个4行7列的网格.

\end{flushleft}
\begin{figure}[!ht]
\centering
\begin{tikzpicture}
	\node at (5,1) [anchor=west]{$A$};
	\node at (0,2) [anchor=south]{$B$};
	\node at (-3,0) [anchor=east]{$C$};
	\node at (0,-2) [anchor=north]{$D$};
	\node at (1,0) [anchor=east]{$E$};
	\node at (3,-2) [anchor=north]{$F$};
	\node at (5,-1) [anchor=west]{$G$};
	\draw[thick](5,1)--(0,-2);
	\draw[thick](0,-2)--(-3,0);
	\draw[thick](-3,0)--(0,2);
	\draw[thick](0,2)--(5,-1);
	\draw[thick](5,-1)--(3,-2);
	\draw[thick](3,-2)--(1,0);
	\draw[thick](1,0)--(5,1);
	\node at (1,-3) [anchor=north]{第一题图};
	
\end{tikzpicture}
\end{figure}
\begin{figure}[!ht]
\centering
\begin{tikzpicture}
	\foreach \x in{0,1,2}
		\draw[thick](\x,0)--(\x,2);
	\foreach \y in{0,1,2}
		\draw[thick](0,\y)--(2,\y);
	\foreach \x in{5.5,6.5,8,11}
		\draw[thick](\x,0)--(\x,1);
	\foreach \x in{3.5,4.5,9,10}
		\draw[thick](\x,0)--(\x,2);
	\draw[thick](3.5,2)--(4.5,2);
	\draw[thick](9,2)--(10,2);
	\draw[thick](3.5,1)--(6.5,1);
	\draw[thick](8,1)--(11,1);
	\draw[thick](3.5,0)--(6.5,0);
	\draw[thick](8,0)--(11,0);
	\draw[thick](2,-1)--(3,-1);
	\draw[thick](5,-1)--(7,-1);
	\draw[thick](8,-1)--(10,-1);
	\foreach \x in{0,4,8}
		\draw[thick](\x,-2)--(\x+3,-2);
	\draw[thick](0,-3)--(3,-3);
	\draw[thick](4,-3)--(6,-3);
	\draw[thick](9,-3)--(11,-3);
	\foreach \x in{2,3,5,6,7,8,9,10}
		\draw[thick](\x,-2)--(\x,-1);
	\foreach \x in{0,1,2,3,4,5,6,9,10,11}
		\draw[thick](\x,-3)--(\x,-2);
	\draw[thick](3.5,-4)--(7.5,-4);
	\draw[thick](3.5,-5)--(7.5,-5);
	\foreach \x in{3.5,4.5,5.5,6.5,7.5}
		\draw[thick](\x,-4)--(\x,-5);
	\node at (5.5,-5.5)[anchor=north]{第二题图};

\end{tikzpicture}
\end{figure}
\end{document}