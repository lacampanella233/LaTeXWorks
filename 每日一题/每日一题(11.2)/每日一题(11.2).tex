\documentclass{article}
\usepackage[UTF8]{ctex}
\usepackage{amsmath,mathtools,geometry,tikz,caption}
\usetikzlibrary{positioning,shapes.geometric}
\geometry{a4paper,scale=0.7}

\title{每日一题(11.2)}
\author{\kaishu 李东宸}

\begin{document}
\maketitle
\textbf{1. }若$a$, $b$满足
\[3a^2+5|b|=7, s=2a^2-3|b|,\]
求$s$的取值范围.\\\par
\textbf{2. }运行程序如图所示, 规定: 从“输入一个值$x$”到“结果是否$>95$”为一次程序操作. 
如果程序运行了3次后停止, 求$x$的取值范围.
\begin{figure}[htbp]
	\centering
	\begin{tikzpicture}[node distance=10pt]
		\node [draw,rounded corners]                    (start)  {开始};
		\node [draw,below=of start]                     (input)  {输入};
		\node [draw,below=of input]                     (x)      {$x$};
		\node [draw,below=of x]                         (times2) {$\times2$};
		\node [draw,below=of times2]                    (+1)     {$+1$};
		\node [draw,diamond,aspect=2,below=of +1]       (>95)    {$>95$?};
		\node [draw,right=20pt of >95]                  (no)     {返回};
		\node [draw,rounded corners,below=15pt of >95]  (end)    {结束};
		
		\draw [->] (start)--(input);
		\draw [->] (input)--(x);
		\draw [->] (x)--(times2);
		\draw [->] (times2)--(+1);
		\draw [->] (+1)--(>95);
		\draw [->] (>95)-- node[left]{是} (end);
		\draw [->] (>95)-- node[above]{否} (no);
		\draw [->] (no)-- (no|-x) -> (x);
	\end{tikzpicture}
	\caption*{\kaishu 第二题图}
\end{figure}
\end{document}