\documentclass{article}
\usepackage[UTF8]{ctex}
\usepackage{amsmath,mathtools,geometry,fancyhdr,amsfonts,amssymb,}
\geometry{a4paper,scale=0.7}
\pagenumbering{arabic}
\pagestyle{fancy}
\fancyhead[L]{励志AQ班}
\fancyhead[R]{每日一题(9.2)答案}
\fancyfoot[C]{\thepage}
\title{每日一题(9.2)答案}
\author{\kaishu 选题: 李政毅\\\kaishu 答案制作:程昊一}

\begin{document}
\maketitle
{\kaishu \textbf{1. }若$m^2=n+2, n^2=m+2$ $(m\neq n)$, 求$m^3-2mn+n^3$的值.\\\par}
\textbf{解(方法一)}\quad 我们使用\textbf{降次}的思想.\par
将前式代入所求的式子,得到
\[m(n+2)-2mn+n(m+2),\]
即
\[2m+2n.\]\par
现在的任务就是求出$m+n$.\par
将条件中的两个式子相减,得到
\[m^2-n^2=n-m,\]
即
\[(m-n)(m+n)=-(m-n).\]
$\because m\neq n$, $\therefore m-n\neq 0$, $\therefore m+n=-1.$\par
所以, 原式$=-2$.\par

\textbf{方法2}\quad 由原式, 得
\[\begin{cases}
	m^2-n=2\\n^2-m=2
\end{cases},\]
所以, 
\begin{align*}
	m^3-2mn+n^3&=\left(m^3-mn\right)+\left(n^3-mn\right)\\
	&=m\left(m^2-n\right)+n\left(m^2-n\right)\\
	&=2m+2n\\
	&=-2.
\end{align*}

{\kaishu \textbf{2. }已知$a+b-c=9, a^2+b^2+c^2=27$, 求$a^{2009}+b^{2009}+c^{2009}$的值. }\\\par
\textbf{分析}\quad 打眼一看这道题的每一个式子都是关于$a,b,c$的轮换式\footnote{所谓轮换式,就是依次交换$a$, $b$, $c$(即$a\to b$, $b\to c$, $c\to a$), 所得到的式子仍然保持不变. 例如, 对于式$ab+bc+ca$, 依次交换$a$, $b$, $c$后, 为$bc+ca+ab$, 与原式相同, 所以$ab+bc+ca$是关于$a$, $b$, $c$的轮换式.}. 可是第一个式子中$c$的符号为负, 所以这道题一定另有解法.\par
\textbf{解}\quad (2)式减去6$\times$(1), 得
\[a^2+b^2+c^2-6a-6b+6c=-27.\]
进行配方, 得
\[\left(a^2-6a+9\right)+\left(b^2-6b+9\right)+\left(c^2+6c+9\right)=0,\]
即
\[(a-3)^2+(b-3)^2+(c+3)^2=0,\]
得$a=b=3$, $c=-3$.\par
代入原式, 得原式=$3^2009$.
\end{document}