\documentclass{article}
\usepackage[UTF8]{ctex}
\usepackage{amsmath,mathtools,enumitem,geometry,fancyhdr}
\pagenumbering{arabic}
\pagestyle{fancy}
\fancyhead[L]{励志AQ班}
\fancyhead[R]{每日一题(7.1)}
\fancyfoot[C]{\thepage}
\geometry{a4paper,scale=0.7}

\title{每日一题(7.1)}
\author{程昊一、李衡岳}
\begin{document}
\maketitle
\textbf{1.}我们定义:如果$a$($a>0$且$a\neq 1$)的$b$次幂等于$N$,那么 $b$称为以$a$为底$N$的对数,记作
\[\log_aN=b.\]
证明:若$a,b>0,a,b\neq 1$,则
\begin{itemize}
	\item[(1)]$\log_ax+\log_ay=\log_axy$\quad$(x,y>0)$;
	\item[(2)]$\log_ax^b=b\log_ax$\quad$(x>0)$;
	\item[(3)]$\log_ax=\dfrac{\log_bx}{\log_ba}$\quad$(x>0)$;
	\item[(4)]$\log_{a^x}b^y=\dfrac{y}{x}\log_ab$\quad$(x\neq 0)$.
\end{itemize}
{\CJKfamily{kai}(程昊一供题)}\\
\par\textbf{2.}有$n$个人,每个人的生日是完全随机且互不相关的.当$n$不小于多少时,存在两个生日相同的人的概率不小于$\dfrac{1}{2}$(假设一天有365天)?\\
{\CJKfamily{kai}(李衡岳供题)}
\end{document}