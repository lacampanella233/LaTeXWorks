\documentclass{article}
\usepackage[UTF8]{ctex}
\usepackage{amsmath,mathtools,geometry,fancyhdr}
\geometry{a4paper,scale=0.7}
\pagenumbering{arabic}
\pagestyle{fancy}
\fancyfoot[C]{\thepage}
\fancyhead[L]{励志AQ班}
\fancyhead[R]{每日一题(10.2)答案}

\title{每日一题(10.2)答案}
\author{\kaishu 选/命题: 门宇翎、程昊一\\\kaishu 答案制作: 程昊一}

\begin{document}
\maketitle
\textbf{1. }{\kaishu 已知$x$, $y$, $z$为实数, 且满足
\[x+2y-5z=3, x-2y-z=-5,\]
求$x^2+y^2+z^2$的最小值.\\
\rightline{(门宇翎供题)}}\\\par
\textbf{分析}\quad 我们看到了3个未知数, 2个等式, 我们可以用其中一个未知数表示另外两个未知数, 然后将需要求的式子写成关于一个未知数的式子, 相对来说就更好处理.\\\par
\textbf{解}\quad 解关于$x$, $y$的方程组
\[\begin{cases}
	x+2y=5z+3\\x-2y=z-5
\end{cases},\]
得
\[\begin{cases}
	x=3z-1\\y=z+2
\end{cases}.\]\par
所以, 原式$=(3z-1)^2+(z+2)^2+z^2=11z^2-2z+5=11\left(z-\dfrac{1}{11}\right)^2+\dfrac{54}{11}.$\par
所以, 原式的最小值为$\dfrac{54}{11}.$\\\par

\textbf{2. }{\kaishu 因式分解:
\[\left(x^2+2y^2\right)^4+64y^8.\]
\rightline{(程昊一命题)}}\\\par
\textbf{分析}\quad 我们先回顾一下$a^4+4b^4$的因式分解的方法:
\begin{align*}
	a^4+4b^4&=a^4+4a^2b^2+4b^4-4a^2b^2\\
	&=\left(a^2+2b^2\right)^2-(2ab)^2\\
	&=\left(a^2-2ab+2b^2\right)\left(a^2+2ab+2b^2\right).
\end{align*}\par
我们在这里进行了一个添项: $4a^2b^2$. 其目的是与前两项配成完全平方, 同时能与配成的完全平方进行平方差.\par
对于这道题, 我们做换元:
\[\begin{cases}
	a=x^2+2y^2\\b=2y^2
\end{cases}, \]
就可以了.\\\par
\textbf{解}\quad 设
\[\begin{cases}
	a=x^2+2y^2\\b=2y^2
\end{cases}, \]
则\begin{align*}
	\text{原式}&=a^4+4b^4\\
	&=\left(a^2-2ab+2b^2\right)\left(a^2+2ab+2b^2\right)\\
	&=\left[\left(x^2+2y^2\right)^2-2\left(x^2+2y^2\right)\left(2y^2\right)+2\left(2y^2\right)^2\right]\left[\left(x^2+2y^2\right)^2+2\left(x^2+2y^2\right)\left(2y^2\right)+2\left(2y^2\right)^2\right]\\
	&=\left(x^4+4y^4+4x^2y^2-4x^2y^2-8y^4+8y^4\right)\left(x^4+4y^4+4x^2y^2+4x^2y^2+8y^4+8y^4\right)\\
	&=\left(x^4+4y^4\right)\left(x^4+20y^4+8x^2y^2\right)\\
	&=\left(x^2-2xy+2y^2\right)\left(x^2+2xy+2y^2\right)\left(x^4+20y^4+8x^2y^2\right).
\end{align*}\par
\textbf{注}\quad 事实上, 我在出这道题时, 核心思想就是利用$a^4+4b^4$的因式分解, 所以构造了上述换元, 使在分解的过程中运用了两次$a^4+4b^4$的因式分解.
\end{document}