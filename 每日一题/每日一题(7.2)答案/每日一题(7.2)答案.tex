\documentclass{article}
\usepackage[UTF8]{ctex}
\usepackage{amsmath,mathtools,geometry,enumitem,fancyhdr}
\geometry{a4paper,scale=0.7}
\pagenumbering{arabic}
\pagestyle{fancy}
\fancyhead[L]{励志AQ班}
\fancyhead[R]{每日一题(7.2)答案}
\fancyfoot[C]{\thepage}
\renewcommand{\footrulewidth}{0.5pt}
\title{每日一题(7.2)答案}
\author{选题:李衡岳、程昊一\\答案制作:程昊一}

\begin{document}
\maketitle
{\CJKfamily{kai}
\textbf{1.}求所有的整数$x,y$,满足方程
\[(x^2-y^2)^2=16y+1.\]
(程昊一供题)}
\par\textbf{分析}\quad 我们发现,等号的左边是四次式,等式的左边是四次式,右边是一次式.因为$x$和$y$是整数,所以关于$x$和$y$的四次式与一次式相等的情况是不多的,因为从大小关系来说,四次式的绝对值一般都大于一次式的绝对值.那么,我们运用\textbf{不等估计}的方法,求出$x$或$y$的范围,这样就可以枚举了.
\par\textbf{解}\quad 由于整数的离散性,以下两个式子必定会有一个成立:
\[\begin{cases}
\left(x^2-y^2\right)^2\ge\left((y+1)^2-y^2\right)^2=(2y+1)^2\\
\left(x^2-y^2\right)^2\ge\left((y-1)^2-y^2\right)^2=(2y-1)^2
\end{cases}\]
即
\[\begin{cases}
4y^2+4y+1\le 16y+1\\
4y^2-4y+1\le 16y+1
\end{cases}\]
至少有一个成立.
即
\[\begin{cases}
0\le y\le 3\\
0\le y\le 5
\end{cases}\]
至少有一个成立.又$y$是整数,所以$y$=0或1或2或3或4或5.
\begin{itemize}
	\item[(1)]$y=0$.此时原方程即为$x^4=1$,又$x$为整数,所以$x=\pm1$.
	\item[(2)]$y=1$.此时原方程即为$(x^2-1)^2=17$,无整数解.
	\item[(3)]$y=2$.此时原方程即为$(x^2-4)^2=33$,无整数解.
	\item[(4)]$y=3$.此时原方程即为$(x^2-9)^2=49$,得$x=\pm4$.
	\item[(5)]$y=4$.此时原方程即为$(x^2-16)^2=65$,无整数解.
	\item[(6)]$y=5$.此时原方程即为$(x^2-25)^2=81$,得$x=\pm4$.
\end{itemize}
\par 综上:原方程的解为$(x,y)=(0,\pm1)$或$(3,\pm4)$或$(5, \pm4)$.\\
{\CJKfamily{kai}
\par\textbf{2.}我们假设有一个村庄,村庄里有很多户人家.每一户人都有一条狗,可能是正常的狗,也可能是疯狗.如果一个人发现自己的狗是疯狗,那么他会在当天晚上把自己的狗击毙.每一个人只可以判断其他人的狗是否为疯狗,每两个人之间也不能互相交流.一天,一位游客向全部的人宣布:“村庄里有疯狗!”当天晚上,没有人击毙自己的狗;第二天亦是如此;第三天有人击毙了自己的狗.问:村庄里有几条疯狗?\\
(李衡岳供题)}
\par\textbf{解}\quad 我们先考虑一种比较简单的情况,即村庄里只有一条疯狗.不妨设疯狗的主人是甲.甲得知了村庄里有一条狗,而且没有看到其他任何一条狗是疯狗,那么毫无疑问,甲一定知道自己的狗是疯狗,那么他会在当天晚上击毙自己的狗,与题设矛盾.
\par 我们再来看有两条疯狗的情况.不妨设两条疯狗的主人为甲和乙.当天,甲和乙都看到了对方的狗是疯狗,所以不能判断自己的狗是否为疯狗,那么第一天晚上没有人把狗击毙,符合题设.第二天,甲看到乙没有击毙自己的疯狗,那么甲就知道,\textbf{除了乙的疯狗,必然存在另外一条疯狗,使得乙无法判断自己的狗.}但是甲只看到了一个乙的疯狗,所以他可以确认自己的狗是疯狗,那么他会在第二天晚上把自己的狗击毙.同理,乙也会把自己的狗在第二天击毙,与题设矛盾.
\par 我们可以按同样的方法分析有三条疯狗的情况.不妨设甲,乙,丙的狗都是疯狗.从甲的角度考虑,若甲自己的狗不是疯狗,那么根据前面的讨论,乙和丙会在第二天晚上击毙自己的狗.可是事实却是前两天晚上并没有人击毙自己的狗.那么这意味着\textbf{除了乙和丙的疯狗,还有其他疯狗.}那么甲这时就可以断定自己的狗是疯狗.于是,甲会在第三天晚上击毙自己的狗.同理,乙和丙也会在第三天晚上击毙自己的狗,与题设吻合.
\par 按照上面的逻辑,我们可以得出:若在第$n$天有人击毙了狗,那么整个村庄就有$n$条疯狗.
\par 综上:村庄里有三条疯狗.
\end{document}