\documentclass[UTF8]{ctexart}
\usepackage{amsmath}
\usepackage{mathtools}
\usepackage{cases}
\usepackage{amssymb}
\usepackage{geometry}
\geometry{a4paper,scale=0.7}
\linespread{1.75}
\title{问题征解(1)答案}
\author{程昊一}
\begin{document}
\maketitle
\begin{flushleft}
%----------%
\textbf{\large 1.证明:素数有无穷多个.}\\
\end{flushleft}
解:假设素数的个数有限,设这些素数为$p_1,p_2,\dots p_n$.\\
考虑$k=p_1p_2p_3\dots p_n+1$,则$p_1\nmid k,p_2\nmid k,\dots p_n\nmid k$.\\
若$k$为素数,则与假设矛盾.\\
若$k$为合数,∵$p_1\nmid k,p_2\nmid k,\dots p_n\nmid k$,∴k有比$p_1,p_2,\dots p_n$更大的素因子,与假设矛盾.\\
∴假设不成立,即素数的个数无限.\\
\\
%----------%
\textbf{\large 2.证明:数列$10001,100010001,1000100010001,\dots$中,每一个数都是合数.}\\
解:(1) $10001=73\times 137$,∴$10001$为合数.\\
(2)$n>1$时,$1\underbrace{00010001\dots 0001}_{\text{共$n$个1}} =10^{4n}+10^{4n-4}+\dots+1=\dfrac{10^{4n+4}-1}{10^4-1}=\dfrac{(10^{2n+2}-1)(10^{2n+2}+1)}{9999}$\\
∵$(10^{2n+2}+1)>(10^{2n+2}-1)>9999$,∴$1\underbrace{00010001\dots 0001}_{\text{共$n$个1}}$为合数.\\
综上,数列$10001,100010001,1000100010001,\dots$中,每一个数都是合数.\\
\\
%----------%
\textbf{\large 3.求所有的正整数$x,y,z$,满足$3^x+4^y=5^z$.}\\
解:对原方程两边同时模3,得到
\[2^z\equiv (-1)^z\equiv 1\pmod{3}\]
所以$2\mid z$,设$z=2z_1$.\\
对原方程两边同时模4,得到
\[3^x\equiv (-1)^x\equiv 1\pmod{4}\]
所以$2\mid x$,设$x=2x_1$.\\
则原方程$\Rightarrow3^{2x_1}+2^{2y}=5^{2z_1}$,即
\[\begin{cases}
3^{2x_1}=(5^{z_1})^2-(2^{y})^2\\
2^{2y}=(5^{z_1})^2-(3^{x_1})^2\\
\end{cases}\]
得到
\begin{numcases}{}
3^{2x_1}=(5^{z_1}-2^{y})(5^{z_1}+2^{y})\label{}\\
2^{2y}=(5^{z_1}-3^{x_1})(5^{z_1}+3^{x_1})\label{}
\end{numcases}
对于$(1)$:\\
∵$(5^{z_1}-2^{y},5^{z_1}+2^{y})=(5^{z_1}-2^{y},2^{y+1}),3\nmid2^{y+1}$,\\
∴$3\nmid(5^{z_1}-2^{y},5^{z_1}+2^{y})$,即$5^{z_1}-2^{y}$与$5^{z_1}+2^{y}$不同时含有素因子3.∴$5^{z_1}-2^{y}$与$5^{z_1}+2^{y}$一个为$3^{2x_1}$,一个为1.\\
又$5^{z_1}-2^{y}<5^{z_1}+2^{y}$,∴有
\[\begin{cases}
5^{z_1}-2^{y}=1\\
5^{z_1}+2^{y}=3^{2x_1}\\
\end{cases}\]
得
\begin{equation}
5^{z_1}=2^{y}+1
\end{equation}
对于$(2)$:\\
∵$(5^{z_1}-3^{x_1},5^{z_1}+3^{x_1})=(5^{z_1}-3^{x_1},2\times3^{x_1}),2\mid5^{z_1}-3^{x_1},2\mid2\times3^{x_1}$而$4\nmid2\times3^{x_1}$\\
∴$(5^{z_1}-3^{x_1},5^{z_1}+3^{x_1})=2$,又$5^{z_1}-3^{x_1}<5^{z_1}+3^{x_1})$,∴有\\
\begin{numcases}{}
5^{z_1}-3^{x_1}=2\label{}\\
5^{z_1}+3^{x_1}=2^{2y-1}\label{}
\end{numcases}
$(4)+(5)$,得
\[2\times 5^{z_1}=2+2^{2y-1}\]
得
\begin{equation}
5^{z_1}=2^{2y-2}+1
\end{equation}
由$(3),(6)$得
\[2^{y}+1=2^{2y-2}+1\]
即
\[y=2y-2\]
得$y=2$.\\
将$y=2$代入$(3)$,得
\[5^{z_1}=5\]
得$z_1=5,z=2$.\\
将$y=2,z=2$代入原方程,立得$x=2$.\\
综上:原方程的正整数解为$x=y=z=2$.\\
\\
%----------%
\textbf{\large 4.已知\[a+b+c=5,a^2+b^2+c^2=15,a^3+b^3+c^3=47\]求\[(a^2+ab+b^2)(b^2+bc+c^2)(c^2+ca+a^2)\]}\\
解:展开$(a^2+ab+b^2)(b^2+bc+c^2)(c^2+ca+a^2)$,得
\[3a^2b^2c^2+\sum\limits_{sym}{a^4b^2}+2\sum\limits_{sym}{a^3b^2c}+\sum\limits_{cyc}{a^4bc}+\sum\limits_{cyc}{a^3b^3}\]
其中$\sum\limits_{sym}$与$\sum\limits_{cyc}$详见脚注.\footnote{
$\sum\limits_{sym}$指将此符号后的式子中的字母随意交换,将得到的所有式子求和,例如$\sum\limits_{sym}{a^2b}=a^2b+b^2c+c^2a+ab^2+bc^2+ca^2$;
$\sum\limits_{cyc}$指将此符号后的式子中的字母依次轮换(例如$a\to b,b\to c,c\to a$),将得到的所有式子求和,例如$\sum\limits_{cyc}{ab}=ab+bc+ca$.}\\
∵$2\sum\limits_{cyc}{ab}=(a+b+c)^2-(a^2+b^2+c^2)=5^2-15=10$\\
∴$\sum\limits_{cyc}{ab}=5$.\\%∑ab=5
∵由欧拉公式,
\begin{align*}
3abc&=(\sum\limits_{cyc}{a^3})-(\sum\limits_{cyc}{a})(\sum\limits_{cyc}{a^2}-\sum\limits_{cyc}{ab})\\
&=47-5\times (15-5)\\
&=-3
\end{align*}
∴$abc=-1$.%abc=-1
\begin{align*}
\sum\limits_{cyc}{a^4bc}&=\sum\limits_{cyc}{(abc\times a^3)}\\
&=abc(\sum\limits_{cyc}{a^3})\\
&=(-1)\times47\\
&=-47
\end{align*}
∴$\sum\limits_{cyc}{a^4bc}=-47$.%∑a^4bc=-47
\begin{align*}
\sum\limits_{cyc}{a^2b^2}&=(\sum\limits_{cyc}{ab})^2-2\sum\limits_{cyc}{(ab\times bc)}\\
&=(\sum\limits_{cyc}{ab})^2-2\sum\limits_{cyc}{(abc\times a)}\\
&=(\sum\limits_{cyc}{ab})^2-2abc\sum\limits_{cyc}{a}\\
&=5^2-2\times (-1)\times 5\\
&=35\end{align*}
∴$\sum\limits_{cyc}{a^2b^2}=35$.%∑a²b²=35
\begin{align*}
\sum\limits_{sym}{a^4b^2}&=(\sum\limits_{cyc}{a^2})(\sum\limits_{cyc}{a^2b^2})-3a^2b^2c^2\\&=15\times 35-3\times(-1)^2\\&=522
\end{align*}
∴$\sum\limits_{sym}{a^4b^2}=522$%∑a^4b^2=522
\begin{align*}
\sum\limits_{cyc}{a^2b}&=(\sum\limits_{cyc}{ab})(\sum\limits_{cyc}{a})-3abc\\
&=5\times 5-3\times(-1)\\
&=28
\end{align*}
∴$\sum\limits_{sym}{a^2b}=28$.%∑a²b=28
\begin{align*}
\sum\limits_{sym}{a^3b^2c}&=\sum\limits_{sym}{(abc\times a^2b)}\\
&=abc\times\sum\limits_{sym}{a^2b}\\
&=(-1)\times 28\\
&=-28
\end{align*}
∴$\sum\limits_{sym}a^3b^2c=-28$.%∑a³b²c=-28
\begin{align*}
\sum\limits_{cyc}{a^3b^3}&=(\sum\limits_{cyc}{ab})(\sum\limits_{cyc}{a^2b^2})-\sum\limits_{sym}{a^3b^2c}\\
&=5\times35-(-28)\\
&=203
\end{align*}
∴$\sum\limits_{cyc}{a^3b^3}=203$\\%∑a³b³=203
∴原式
\begin{align*}
&=3a^2b^2c^2+\sum\limits_{sym}{a^4b^2}+2\sum\limits_{sym}{a^3b^2c}+\sum\limits_{cyc}{a^4bc}+\sum\limits_{cyc}{a^3b^3}\\
&=3\times(-1)^2+522+2\times(-28)+(-47)+203\\
&=625
\end{align*}
综上:原式$=625$.


\end{document}

